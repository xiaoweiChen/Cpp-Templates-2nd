
\documentclass[11pt,a4paper,UTF8]{book}

\usepackage[T1]{fontenc}
\usepackage[utf8]{inputenc}
\usepackage{authblk}

\usepackage{fontspec}                  %引入字体设置宏包
\setmainfont{Times New Roman}             %设置英文正文字体
% Courier New
% Book Antique
\setsansfont{Arial}                    %英文无衬线字体
\setmonofont{Courier New}              %英文等宽字体

\usepackage{ctex} %导入中文包
%\usepackage{ulem}
\usepackage{tocvsec2}
\usepackage{verbatim}

\usepackage{tabularx}
\usepackage{booktabs} 
\usepackage{multirow}
\usepackage{bbding}
\usepackage{float}
\usepackage{xspace}
\usepackage[none]{hyphenat}

\usepackage{graphicx}
\usepackage{subfigure}
\usepackage{pifont}

\usepackage{hyperref}  %制作pdf的目录
\usepackage{subfiles} %使用多文件方式进行

\usepackage{geometry} %设置页边距的包
\geometry{left=2.5cm,right=2cm,top=2.54cm,bottom=2.54cm} %设置书籍的页边距

\usepackage{url}
\hypersetup{hidelinks, %去红框
	colorlinks=true,
	allcolors=black,
	pdfstartview=Fit,
	breaklinks=true
}

% 调整itemlist中的行间距
\usepackage{enumitem}
\setenumerate[1]{itemsep=0pt,partopsep=0pt,parsep=\parskip,topsep=5pt}
\setitemize[1]{itemsep=0pt,partopsep=0pt,parsep=\parskip,topsep=5pt}
\setdescription{itemsep=0pt,partopsep=0pt,parsep=\parskip,topsep=5pt}

% 超链接样式设置
\usepackage{hyperref}
\hypersetup{
	colorlinks=true,
	linkcolor=blue,
	filecolor=blue,
	urlcolor=blue,
	citecolor=cyan,
}

\usepackage{indentfirst}

\usepackage{listings}
\usepackage[usenames,dvipsnames,svgnames, x11names]{xcolor}

\usepackage[most]{tcolorbox}

%展示代码
\definecolor{mygreen}{rgb}{0,0.6,0}
\definecolor{mygray}{rgb}{0.5,0.5,0.5}
\definecolor{mymauve}{rgb}{0.58,0,0.82}
\definecolor{keywordcolor}{rgb}{0.8,0.1,0.5}
\definecolor{webgreen}{rgb}{0,.5,0}
\definecolor{bgcolor}{rgb}{0.92,0.92,0.92}

%定义CMake
\lstdefinelanguage{CMake}
{morekeywords={
		cmake\_minimum\_required,
		project,
		add\_executable,
		add\_library,
		target\_link\_libraries,
		cmake\_parse\_arguments,
		cmake\_language,
		set, unset,
		option,
		string,
		list,
		math,
		message,
		if, elseif, else, endif,
		mark\_as\_advanced,
		foreach, endforeach,
		while, endwhile,
		add\_subdirectory, include, return, include\_gurad,
		function, endfunction,
		macro, endmacro,
		find\_package,
		cmake\_push\_check\_state,
		cmake\_pop\_check\_state,
		cmake\_reset\_check\_state,
		add\_test,
		set\_tests\_properties, 
		check\_c\_source\_runs,
		check\_cxx\_source\_runs,
		check\_fortran\_source\_runs,
		check\_source\_runs,
		check\_compiler\_flag,
		check\_c\_compiler\_flag,
		check\_cxx\_compiler\_flag,
		check\_fortran\_compiler\_flag,
		check\_symbol\_exists,
		check\_cxx\_symbol\_exists,
		check\_linker\_flag,
		cmake\_policy,
		set\_property,
		get\_property,
		define\_property,
		get\_cmake\_property,
		set\_cmake\_property,
		set\_target\_properties,
		get\_target\_property,
		set\_directory\_properties,
		get\_directory\_property,
		set\_source\_files\_properties,
		get\_source\_file\_property,
		set\_tests\_properties,
		get\_tests\_property,
		get\_test\_property,
		cmake\_print\_properties,
		cmake\_print\_variables,
		variable\_watch,
		include\_guard,
		target\_link\_options,
		target\_compile\_definitions,
		target\_compile\_options,
		include\_directories,
		add\_definitions,
		remove\_definitions,
		add\_compile\_definitions,
		add\_compile\_options,
		link\_libraries,
		link\_directories,
		add\_link\_options,
		target\_include\_directories,
		target\_compile\_features,
		add\_custom\_command,
		add\_custom\_target,
		execute\_process,
		cmake\_path,
		get\_filename\_component,
		file,
		configure\_file,
		generate\_export\_header,
		export,
		find\_file,
		find\_library,
		find\_package,
		find\_program,
		pkg\_check\_modules,
		pkg\_search\_module,
		pkg\_get\_variable,
		add\_test,
		enable\_testing,
		set\_tests\_properties,
		site\_name,
		ctest\_empty\_binary\_directory,
		ctest\_start,
		ctest\_configure,
		ctest\_submit,
		ctest\_build,
		ctest\_memcheck,
		ctest\_upload,
		ctest\_test,
		gtest\_add\_tests,
		gtest\_discover\_tests,
		install,
		write\_basic\_package\_version\_file,
		configure\_package\_config\_file,
		cpack\_add\_component,
		cpack\_add\_install\_type,
		cpack\_add\_component\_group,
		ExternalProject\_Add,
		ExternalProject\_Add\_StepDependencies,
		ExternalProject\_Get\_Property,
		ExternalProject\_Add\_Step,
		FetchContent\_Declare,
		FetchContent\_GetProperties,
		FetchContent\_Populate,
		source\_group,
		target\_precompile\_headers,
		qt5\_wrap\_cpp,
		qt5\_wrap\_ui,
		qt5\_add\_resources,
		qt5\_add\_big\_resources,
		qt5\_add\_binary\_resources,
		qt5\_add\_translation,
		qt5\_create\_translation,
		compile\_definitions,
		add\_llvm\_component\_library,
		add\_llvm\_tool,
		llvm\_multisource,
		llvm\_test\_data,
		doxygen\_add\_docs,
	}, %定义关键字
	sensitive=false, %是否大小写敏感
	morecomment=[l]{\#},
	morestring=[b]",
	morestring=[d]',
}

\lstdefinestyle{styleCXX}{
	language = C++,  
	backgroundcolor=\color{blue!3!white}, 
	%basicstyle = \footnotesize,  
	basicstyle      =   \zihao{-5}\ttfamily,
	numberstyle     =   \zihao{-5}\ttfamily,   
	%breakatwhitespace = false,    
	basewidth       =   0.5em,    
	breaklines = true,                 
	captionpos = b,                    
	commentstyle = \color{mygray}\bfseries,
	%extendedchars = false,             
	frame =shadowbox, 
	framerule=0.5pt,
	%frameround = fttt,
	keepspaces=true,
	keywordstyle=\color{blue}\bfseries, % keyword style
	otherkeywords={string}, 
	numbers=left, 
	numbersep=5pt,
	numberstyle=\tiny\color{mygray},
	rulecolor=\color{black},         
	%showspaces=false,  
	%showstringspaces=false, 
	%showtabs=false,    
	%stepnumber=1,         
	stringstyle=\color{mymauve},        % string literal style
	tabsize=2,          
	columns         =   fixed,
	flexiblecolumns,                   
}


\lstdefinestyle{styleCMake}{
	language=CMake,
	backgroundcolor=\color{blue!3!white}, 
	basicstyle=\tt, 
	breakatwhitespace = false,
	breaklines = true,
	captionpos = b,
	commentstyle = \color{mygray}\bfseries, 
	extendedchars =false,             
	frame=shadowbox, 
	tabsize=2,
	framerule=0.5pt,
	keepspaces=true,
	keywordstyle=\color{blue}\bfseries, % keyword style
	otherkeywords={string}, 
	rulecolor=\color{black},
	showspaces=false,
	showstringspaces=false,
	showtabs=false,
	stepnumber=1,
	stringstyle=\color{purple},        % string literal style
}

\lstdefinestyle{stylePython}{
	language        =   Python, % 语言选Python
	backgroundcolor=\color{blue!3!white}, 
	basicstyle      =   \zihao{-5}\ttfamily,
	numberstyle     =   \zihao{-5}\ttfamily,
	keywordstyle    =   \color{blue},
	keywordstyle    =   [2] \color{teal},
	stringstyle     =   \color{magenta},
	commentstyle    =   \color{red}\ttfamily,
	frame = shadowbox, 
	breaklines      =   true,   % 自动换行,建议不要写太长的行
	columns         =   fixed,  % 如果不加这一句,字间距就不固定,很丑,必须加
	basewidth       =   0.5em,
	%basicstyle          =   \sffamily,          % 基本代码风格
	%keywordstyle        =   \bfseries,          % 关键字风格
	%commentstyle        =   \rmfamily\itshape,  % 注释的风格,斜体
	%stringstyle         =   \ttfamily,  % 字符串风格
	flexiblecolumns,                % 别问为什么,加上这个
	%numbers             =   left,   % 行号的位置在左边
	showspaces          =   false,  % 是否显示空格,显示了有点乱,所以不现实了
	numberstyle         =   \zihao{-5}\ttfamily,    % 行号的样式,小五号,tt等宽字体
	showstringspaces    =   false,
	captionpos          =   t,      % 这段代码的名字所呈现的位置,t指的是top上面
	frame               =   lrtb,   % 显示边框
	tabsize=2,  
}

\tcbset{
	commandshell/.style={
		listing only,
		colback=black!75!white,
		colupper=white,
		lowerbox=ignored,
		listing options={
			language={bash},
			basicstyle=\ttfamily,
			columns = fixed,
			flexiblecolumns
		}
}}

\usepackage{tikz}

% URL 正确换行
% https://liam.page/2017/05/17/help-the-url-command-from-hyperref-to-break-at-line-wrapping-point/
\makeatletter
\def\UrlAlphabet{%
	\do\a\do\b\do\c\do\d\do\e\do\f\do\g\do\h\do\i\do\j%
	\do\k\do\l\do\m\do\n\do\o\do\p\do\q\do\r\do\s\do\t%
	\do\u\do\v\do\w\do\x\do\y\do\z\do\A\do\B\do\C\do\D%
	\do\E\do\F\do\G\do\H\do\I\do\J\do\K\do\L\do\M\do\N%
	\do\O\do\P\do\Q\do\R\do\S\do\T\do\U\do\V\do\W\do\X%
	\do\Y\do\Z}
\def\UrlDigits{\do\1\do\2\do\3\do\4\do\5\do\6\do\7\do\8\do\9\do\0}
\g@addto@macro{\UrlBreaks}{\UrlOrds}
\g@addto@macro{\UrlBreaks}{\UrlAlphabet}
\g@addto@macro{\UrlBreaks}{\UrlDigits}
\makeatother

% enable subsubsubsection
% from https://tex.stackexchange.com/questions/274212/correct-hierarchy-levels-of-pdf-bookmarks-for-custom-section-subsubsubsection
\usepackage[depth=3]{bookmark}
\setcounter{secnumdepth}{3}
\setcounter{tocdepth}{4}
\hypersetup{bookmarksdepth=4}

\makeatletter

\newcommand{\toclevel@subsubsubsection}{4}
\newcounter{subsubsubsection}[subsubsection]

\renewcommand{\thesubsubsubsection}{\thesubsubsection.\arabic{subsubsubsection}}

\newcommand{\subsubsubsection}{\@startsection{subsubsubsection}{4}{\z@}%
	{-3.25ex\@plus -1ex \@minus -.2ex}%
	{1.5ex \@plus .2ex}%
	{\normalfont\normalsize\bf\bfseries}}

\newcommand*{\l@subsubsubsection}{\@dottedtocline{4}{11em}{5em}}  

\newcommand{\subsubsubsectionmark}[1]{}
\makeatother

\begin{document}
\begin{sloppypar} %latex中一行文字出现溢出问题的解决方法
	%\maketitle
	
	\begin{center}
		\thispagestyle{empty}
		%\includegraphics[width=\textwidth,height=\textheight,keepaspectratio]{cover.jpg}
		\begin{tikzpicture}[remember picture, overlay, inner sep=0pt]
			\node at (current page.center) 
			{\includegraphics[width=\paperwidth, keepaspectratio=false]{cover.jpg}};
		\end{tikzpicture}
		\newpage
		\thispagestyle{empty}
		\huge
		\textbf{C++ Templates} 
		\\[9pt]
		\normalsize
		Second Edition
		\\[10pt]
		\normalsize 
		作者: David Vandevoorde,Nicolai M. Josuttis,Douglas Gregor
		\\[8pt]
		\normalsize
		译者:陈晓伟
	\end{center}
	
	\hspace*{\fill} \\ %插入空行
	\noindent\textbf{本书概述}
	
	模板是C++中一个强大的特性,但对模板的误解,并未随着C++语言和开发社区的发展而消弭,从而无法使模板无法发挥其全力。本书的三位作者,同时作为C++专家,展示了如何使用现代模板来构建干净、快捷、高效、容易维护的软件。
	
	第二版对C++11、C++14和C++17标准进行了更新,对改进模板或与模板交互的特性进行了解释,包括可变参数模板、泛型Lambda、类模板参数演绎、编译时if、转发引用和用户定义文字。还深入研究了一些基本的语言概念(比如值类别),并包含了所有标准类型特征。
	
	本书从基本概念和相关语言特征开始,其余部分作为参考。先关注语言,再是编码、高级应用程序和复杂的惯用法。过程中,示例清楚地说明了抽象概念,并演示了模板的最佳实践。
	
	\hspace*{\fill} \\ %插入空行
	\noindent\textbf{关键特性}
	\begin{itemize}
		\item 准确理解模板的行为,避免陷阱
		\item 使用模板编写有效、灵活、可维护的软件
		\item 掌握有效的习语和技巧
		\item 保持性能或安全的情况下重用源码
		\item C++标准库中的泛型编程
		\item 预览即将发布的“概念"特性
	\end{itemize}
	
	\hspace*{\fill} \\ %插入空行
	\noindent\textbf{作者简介}
	
	\textbf{David Vandevoorde}在20世纪80年代后期开始用C++编程。从伦斯勒理工学院获得博士学位后,成为惠普C++编译器团队的技术负责人。1999年,加入了爱迪生设计集团(EDG),该集团的C++编译器技术是业界领先的。他是C++标准委员会的活跃成员,也是comp.lang.c++新闻组的主持人(参与创办)。也是《C++ Solutions》的作者,该书是《C++ Programming Language, 3rd Edition》的配套书籍。
	
	\textbf{Nicolai M. Josuttis}因其畅销的标准书籍《The C++ Standard Library - A Tutorial and Reference》而闻名于世,是一名独立技术顾问,为电信、交通、金融和制造业设计面向对象的软件。也是C++标准委员会的活跃成员,也是System Bauhaus的合伙人,System Bauhaus是一个由面向对象系统开发专家组成的德国团体。Josuttis还写过其他几本关于面向对象编程和C++的书。
	
	\textbf{Douglas Gregor}是苹果公司的高级Swift/C++/Objective-C编译工程师,拥有伦斯勒理工学院的计算机科学博士学位,并在印第安纳大学从事博士后工作。
	
	\hspace*{\fill} \\ %插入空行
	\noindent\textbf{本书相关}
	\begin{itemize}
		\item Github地址:\\\url{https://github.com/xiaoweiChen/Cpp-Templates-2nd}
		\item 开源翻译
		\begin{itemize}
			\item[-]
			\url{https://github.com/Walton1128/CPP-Templates-2nd--}
			\item[-]
			\url{https://github.com/r00tk1ts/cpp-templates-2nd}
			\item[-]
			\url{https://www.zhihu.com/column/c_1397602018500890624}
		\end{itemize}
	\end{itemize}
	\newpage
	
	%前言
	\pagestyle{empty}
	\subfile{content/preface.tex}
	\newpage
	
	\pagestyle{empty}
	\subfile{content/Acknowledgments-for-the-Second.tex}
	\newpage
	
	\pagestyle{empty}
	\subfile{content/Acknowledgments-for-the-First.tex}
	\newpage
	
	\pagestyle{empty}
	\subfile{content/About-This-Book.tex}
	\newpage
	
	\tableofcontents
	\newpage

	\setsecnumdepth{section}
	
	\color{white}
	\section*{\zihao{1}第一部分:基础知识}
	\pagecolor{mygray}
	\addcontentsline{toc}{section}{第一部分:基础知识}
	\textbf{\subfile{content/1/Section.tex}}
	\newpage
	\color{black}
	\pagecolor{white}

	\subsection*{\zihao{2} 第1章\hspace{0.5cm}函数模板}
	\addcontentsline{toc}{subsection}{第1章\hspace{0.5cm}函数模板}
	\subfile{content/1/chapter1/0.tex}
	
	\subsubsection*{\zihao{3} 1.1.\hspace{0.2cm}初识函数模板}
	\addcontentsline{toc}{subsubsection}{1.1.\hspace{0.2cm}初识函数模板}
	\subfile{content/1/chapter1/1.tex}
	
	\subsubsection*{\zihao{3} 1.2.\hspace{0.2cm}模板参数推导}
	\addcontentsline{toc}{subsubsection}{1.2.\hspace{0.2cm}T模板参数推导}
	\subfile{content/1/chapter1/2.tex}
	
	\subsubsection*{\zihao{3} 1.3.\hspace{0.2cm}多模板参数}
	\addcontentsline{toc}{subsubsection}{1.3.\hspace{0.2cm}多模板参数}
	\subfile{content/1/chapter1/3.tex}
	
	\subsubsection*{\zihao{3} 1.4.\hspace{0.2cm}默认模板参数}
	\addcontentsline{toc}{subsubsection}{1.4.\hspace{0.2cm}默认模板参数}
	\subfile{content/1/chapter1/4.tex}
	
	\subsubsection*{\zihao{3} 1.5.\hspace{0.2cm}重载函数模板}
	\addcontentsline{toc}{subsubsection}{1.5.\hspace{0.2cm}重载函数模板}
	\subfile{content/1/chapter1/5.tex}
	
	\subsubsection*{\zihao{3} 1.6.\hspace{0.2cm}常见问题}
	\addcontentsline{toc}{subsubsection}{1.6.\hspace{0.2cm}常见问题}
	\subfile{content/1/chapter1/6.tex}
	
	\subsubsection*{\zihao{3} 1.7.\hspace{0.2cm}总结}
	\addcontentsline{toc}{subsubsection}{1.7.\hspace{0.2cm}总结}
	\subfile{content/1/chapter1/7.tex}
	\newpage
	
	\subsection*{\zihao{2} 第2章\hspace{0.5cm}类模板}
	\addcontentsline{toc}{subsection}{第2章\hspace{0.5cm}类模板}
	\subfile{content/1/chapter2/0.tex}
	
	\subsubsection*{\zihao{3} 2.1.\hspace{0.2cm}实现栈类模板}
	\addcontentsline{toc}{subsubsection}{2.1.\hspace{0.2cm}实现栈类模板}
	\subfile{content/1/chapter2/1.tex}
	
	\subsubsection*{\zihao{3} 2.2.\hspace{0.2cm}使用栈类模板}
	\addcontentsline{toc}{subsubsection}{2.2.\hspace{0.2cm}使用栈类模板}
	\subfile{content/1/chapter2/2.tex}
	
	\subsubsection*{\zihao{3} 2.3.\hspace{0.2cm}部分使用栈类模板}
	\addcontentsline{toc}{subsubsection}{2.3.\hspace{0.2cm}部分使用栈类模板}
	\subfile{content/1/chapter2/3.tex}
	
	\subsubsection*{\zihao{3} 2.4.\hspace{0.2cm}友元}
	\addcontentsline{toc}{subsubsection}{2.4.\hspace{0.2cm}友元}
	\subfile{content/1/chapter2/4.tex}
	
	\subsubsection*{\zihao{3} 2.5.\hspace{0.2cm}类模板的特化}
	\addcontentsline{toc}{subsubsection}{2.5.\hspace{0.2cm}类模板的特化}
	\subfile{content/1/chapter2/5.tex}
	
	\subsubsection*{\zihao{3} 2.6.\hspace{0.2cm}偏特化}
	\addcontentsline{toc}{subsubsection}{2.6.\hspace{0.2cm}偏特化}
	\subfile{content/1/chapter2/6.tex}
	
	\subsubsection*{\zihao{3} 2.7.\hspace{0.2cm}类模板的默认参数}
	\addcontentsline{toc}{subsubsection}{2.7.\hspace{0.2cm}类模板的默认参数}
	\subfile{content/1/chapter2/7.tex}
	
	\subsubsection*{\zihao{3} 2.8.\hspace{0.2cm}类型别名}
	\addcontentsline{toc}{subsubsection}{2.8.\hspace{0.2cm}类型别名}
	\subfile{content/1/chapter2/8.tex}
	
	\subsubsection*{\zihao{3} 2.9.\hspace{0.2cm}类模板参数推断}
	\addcontentsline{toc}{subsubsection}{2.9.\hspace{0.2cm}类模板参数推断}
	\subfile{content/1/chapter2/9.tex}
	
	\subsubsection*{\zihao{3} 2.10.\hspace{0.2cm}模板聚合}
	\addcontentsline{toc}{subsubsection}{2.10.\hspace{0.2cm}模板聚合}
	\subfile{content/1/chapter2/10.tex}
	
	\subsubsection*{\zihao{3} 2.11.\hspace{0.2cm}总结}
	\addcontentsline{toc}{subsubsection}{2.11.\hspace{0.2cm}总结}
	\subfile{content/1/chapter2/11.tex}
	\newpage
	
	\subsection*{\zihao{2} 第3章\hspace{0.5cm}非类型模板参数}
	\addcontentsline{toc}{subsection}{第3章\hspace{0.5cm}非类型模板参数}
	\subfile{content/1/chapter3/0.tex}
	
	\subsubsection*{\zihao{3} 3.1.\hspace{0.2cm}非类型类模板参数}
	\addcontentsline{toc}{subsubsection}{3.1.\hspace{0.2cm}非类型类模板参数}
	\subfile{content/1/chapter3/1.tex}
	
	\subsubsection*{\zihao{3} 3.2.\hspace{0.2cm}非类型函数模板参数}
	\addcontentsline{toc}{subsubsection}{3.2.\hspace{0.2cm}非类型函数模板参数}
	\subfile{content/1/chapter3/2.tex}
	
	\subsubsection*{\zihao{3} 3.3.\hspace{0.2cm}非类型模板参数的限制}
	\addcontentsline{toc}{subsubsection}{3.3.\hspace{0.2cm}非类型模板参数的限制}
	\subfile{content/1/chapter3/3.tex}
	
	\subsubsection*{\zihao{3} 3.4.\hspace{0.2cm}模板参数类型auto}
	\addcontentsline{toc}{subsubsection}{3.4.\hspace{0.2cm}模板参数类型auto}
	\subfile{content/1/chapter3/4.tex}
	
	\subsubsection*{\zihao{3} 3.5.\hspace{0.2cm}总结}
	\addcontentsline{toc}{subsubsection}{3.5.\hspace{0.2cm}总结}
	\subfile{content/1/chapter3/5.tex}
	\newpage
	
	\subsection*{\zihao{2} 第4章\hspace{0.5cm}可变参数模板}
	\addcontentsline{toc}{subsection}{第4章\hspace{0.5cm}可变参数模板}
	\subfile{content/1/chapter4/0.tex}
	
	\subsubsection*{\zihao{3} 4.1.\hspace{0.2cm}介绍可变参数模板}
	\addcontentsline{toc}{subsubsection}{4.1.\hspace{0.2cm}介绍可变参数模板}
	\subfile{content/1/chapter4/1.tex}
	
	\subsubsection*{\zihao{3} 4.2.\hspace{0.2cm}折叠表达式}
	\addcontentsline{toc}{subsubsection}{4.2.\hspace{0.2cm}折叠表达式}
	\subfile{content/1/chapter4/2.tex}
	
	\subsubsection*{\zihao{3} 4.3.\hspace{0.2cm}可变参数模板的应用}
	\addcontentsline{toc}{subsubsection}{4.3.\hspace{0.2cm}可变参数模板的应用}
	\subfile{content/1/chapter4/3.tex}
	
	\subsubsection*{\zihao{3} 4.4.\hspace{0.2cm}可变参数类模板和表达式}
	\addcontentsline{toc}{subsubsection}{4.4.\hspace{0.2cm}可变参数类模板和表达式}
	\subfile{content/1/chapter4/4.tex}
	
	\subsubsection*{\zihao{3} 4.5.\hspace{0.2cm}总结}
	\addcontentsline{toc}{subsubsection}{4.5.\hspace{0.2cm}总结}
	\subfile{content/1/chapter4/5.tex}
	\newpage
	
	\subsection*{\zihao{2} 第5章\hspace{0.5cm}基础技巧}
	\addcontentsline{toc}{subsection}{第5章\hspace{0.5cm}基础技巧}
	\subfile{content/1/chapter5/0.tex}
	
	\subsubsection*{\zihao{3} 5.1.\hspace{0.2cm}关键字typename}
	\addcontentsline{toc}{subsubsection}{5.1.\hspace{0.2cm}关键字typename}
	\subfile{content/1/chapter5/1.tex}
	
	\subsubsection*{\zihao{3} 5.2.\hspace{0.2cm}零值初始化}
	\addcontentsline{toc}{subsubsection}{5.2.\hspace{0.2cm}零值初始化}
	\subfile{content/1/chapter5/2.tex}
	
	\subsubsection*{\zihao{3} 5.3.\hspace{0.2cm}使用this\texttt{->}}
	\addcontentsline{toc}{subsubsection}{5.3.\hspace{0.2cm}使用this\texttt{->}}
	\subfile{content/1/chapter5/3.tex}
	
	\subsubsection*{\zihao{3} 5.4.\hspace{0.2cm}原始数组和字符串字面量的模板}
	\addcontentsline{toc}{subsubsection}{5.4.\hspace{0.2cm}原始数组和字符串字面量的模板}
	\subfile{content/1/chapter5/4.tex}
	
	\subsubsection*{\zihao{3} 5.5.\hspace{0.2cm}成员模板}
	\addcontentsline{toc}{subsubsection}{5.5.\hspace{0.2cm}成员模板}
	\subfile{content/1/chapter5/5.tex}
	
	\subsubsection*{\zihao{3} 5.6.\hspace{0.2cm}变量模板}
	\addcontentsline{toc}{subsubsection}{5.6.\hspace{0.2cm}变量模板}
	\subfile{content/1/chapter5/6.tex}
	
	\subsubsection*{\zihao{3} 5.7.\hspace{0.2cm}双重模板参数}
	\addcontentsline{toc}{subsubsection}{5.7.\hspace{0.2cm}双重模板参数}
	\subfile{content/1/chapter5/7.tex}
	
	\subsubsection*{\zihao{3} 5.8.\hspace{0.2cm}总结}
	\addcontentsline{toc}{subsubsection}{5.8.\hspace{0.2cm}总结}
	\subfile{content/1/chapter5/8.tex}
	\newpage
	
	\subsection*{\zihao{2} 第6章\hspace{0.5cm}移动语义与enable\_if<>}
	\addcontentsline{toc}{subsection}{第6章\hspace{0.5cm}移动语义与enable\_if<>}
	\subfile{content/1/chapter6/0.tex}
	
	\subsubsection*{\zihao{3} 6.1.\hspace{0.2cm}完美转发}
	\addcontentsline{toc}{subsubsection}{6.1.\hspace{0.2cm}完美转发}
	\subfile{content/1/chapter6/1.tex}
	
	\subsubsection*{\zihao{3} 6.2.\hspace{0.2cm}特殊成员函数模板}
	\addcontentsline{toc}{subsubsection}{6.2.\hspace{0.2cm}特殊成员函数模板}
	\subfile{content/1/chapter6/2.tex}
	
	\subsubsection*{\zihao{3} 6.3.\hspace{0.2cm}使用enable\_if<>禁用模板}
	\addcontentsline{toc}{subsubsection}{6.3.\hspace{0.2cm}使用enable\_if<>禁用模板}
	\subfile{content/1/chapter6/3.tex}
	
	\subsubsection*{\zihao{3} 6.4.\hspace{0.2cm}使用enable\_if<>}
	\addcontentsline{toc}{subsubsection}{6.4.\hspace{0.2cm}使用enable\_if<>}
	\subfile{content/1/chapter6/4.tex}
	
	\subsubsection*{\zihao{3} 6.5.\hspace{0.2cm}使用概念简化enable\_if<>表达式}
	\addcontentsline{toc}{subsubsection}{6.5.\hspace{0.2cm}使用概念简化enable\_if<>表达式}
	\subfile{content/1/chapter6/5.tex}
	
	\subsubsection*{\zihao{3} 6.6.\hspace{0.2cm}总结}
	\addcontentsline{toc}{subsubsection}{6.6.\hspace{0.2cm}总结}
	\subfile{content/1/chapter6/6.tex}
	\newpage
	
	\subsection*{\zihao{2} 第7章\hspace{0.5cm}使用值还是引用?}
	\addcontentsline{toc}{subsection}{第7章\hspace{0.5cm}使用值还是引用?}
	\subfile{content/1/chapter7/0.tex}
	
	\subsubsection*{\zihao{3} 7.1.\hspace{0.2cm}按值传递}
	\addcontentsline{toc}{subsubsection}{7.1.\hspace{0.2cm}按值传递}
	\subfile{content/1/chapter7/1.tex}
	
	\subsubsection*{\zihao{3} 7.2.\hspace{0.2cm}按引用传递}
	\addcontentsline{toc}{subsubsection}{7.2.\hspace{0.2cm}按引用传递}
	\subfile{content/1/chapter7/2.tex}
	
	\subsubsection*{\zihao{3} 7.3.\hspace{0.2cm}使用std::ref()和std::cref()}
	\addcontentsline{toc}{subsubsection}{7.3.\hspace{0.2cm}使用std::ref()和std::cref()}
	\subfile{content/1/chapter7/3.tex}
	
	\subsubsection*{\zihao{3} 7.4.\hspace{0.2cm}处理字符串字面值和数组}
	\addcontentsline{toc}{subsubsection}{7.4.\hspace{0.2cm}处理字符串字面值和数组}
	\subfile{content/1/chapter7/4.tex}
	
	\subsubsection*{\zihao{3} 7.5.\hspace{0.2cm}处理返回值}
	\addcontentsline{toc}{subsubsection}{7.5.\hspace{0.2cm}处理返回值}
	\subfile{content/1/chapter7/5.tex}
	
	\subsubsection*{\zihao{3} 7.6.\hspace{0.2cm}推荐的模板参数声明}
	\addcontentsline{toc}{subsubsection}{7.6.\hspace{0.2cm}推荐的模板参数声明}
	\subfile{content/1/chapter7/6.tex}
	
	\subsubsection*{\zihao{3} 7.7.\hspace{0.2cm}总结}
	\addcontentsline{toc}{subsubsection}{7.7.\hspace{0.2cm}总结}
	\subfile{content/1/chapter7/7.tex}
	\newpage
	
	\subsection*{\zihao{2} 第8章\hspace{0.5cm}编译时编程}
	\addcontentsline{toc}{subsection}{第8章\hspace{0.5cm}编译时编程}
	\subfile{content/1/chapter8/0.tex}
	
	\subsubsection*{\zihao{3} 8.1.\hspace{0.2cm}模板元编程}
	\addcontentsline{toc}{subsubsection}{8.1.\hspace{0.2cm}模板元编程}
	\subfile{content/1/chapter8/1.tex}
	
	\subsubsection*{\zihao{3} 8.2.\hspace{0.2cm}使用constexpr进行计算}
	\addcontentsline{toc}{subsubsection}{8.2.\hspace{0.2cm}使用constexpr进行计算}
	\subfile{content/1/chapter8/2.tex}
	
	\subsubsection*{\zihao{3} 8.3.\hspace{0.2cm}使用偏特化的执行路径选择}
	\addcontentsline{toc}{subsubsection}{8.3.\hspace{0.2cm}使用偏特化的执行路径选择}
	\subfile{content/1/chapter8/3.tex}
	
	\subsubsection*{\zihao{3} 8.4.\hspace{0.2cm}SFINAE(替换失败不为过)}
	\addcontentsline{toc}{subsubsection}{8.4.\hspace{0.2cm}SFINAE(替换失败不为过)}
	\subfile{content/1/chapter8/4.tex}
	
	\subsubsection*{\zihao{3} 8.5.\hspace{0.2cm}编译时if}
	\addcontentsline{toc}{subsubsection}{8.5.\hspace{0.2cm}编译时if}
	\subfile{content/1/chapter8/5.tex}
	
	\subsubsection*{\zihao{3} 8.6.\hspace{0.2cm}总结}
	\addcontentsline{toc}{subsubsection}{8.6.\hspace{0.2cm}总结}
	\subfile{content/1/chapter8/6.tex}
	\newpage
	
	\subsection*{\zihao{2} 第9章\hspace{0.5cm}实际使用模板}
	\addcontentsline{toc}{subsection}{第9章\hspace{0.5cm}实际使用模板}
	\subfile{content/1/chapter9/0.tex}
	
	\subsubsection*{\zihao{3} 9.1.\hspace{0.2cm}包含模型}
	\addcontentsline{toc}{subsubsection}{9.1.\hspace{0.2cm}包含模型}
	\subfile{content/1/chapter9/1.tex}
	
	\subsubsection*{\zihao{3} 9.2.\hspace{0.2cm}模板和内联}
	\addcontentsline{toc}{subsubsection}{9.2.\hspace{0.2cm}模板和内联}
	\subfile{content/1/chapter9/2.tex}
	
	\subsubsection*{\zihao{3} 9.3.\hspace{0.2cm}预编译头文件}
	\addcontentsline{toc}{subsubsection}{9.3.\hspace{0.2cm}预编译头文件}	
	\subfile{content/1/chapter9/3.tex}
	
	\subsubsection*{\zihao{3} 9.4.\hspace{0.2cm}解析编译错误}
	\addcontentsline{toc}{subsubsection}{9.4.\hspace{0.2cm}解析编译错误}
	\subfile{content/1/chapter9/4.tex}
	
	\subsubsection*{\zihao{3} 9.5.\hspace{0.2cm}后记}
	\addcontentsline{toc}{subsubsection}{9.5.\hspace{0.2cm}后记}
	\subfile{content/1/chapter9/5.tex}
	
	\subsubsection*{\zihao{3} 9.6.\hspace{0.2cm}总结}
	\addcontentsline{toc}{subsubsection}{9.6.\hspace{0.2cm}总结}
	\subfile{content/1/chapter9/6.tex}
	\newpage
	
	\subsection*{\zihao{2} 第10章\hspace{0.5cm}基本模板的术语}
	\addcontentsline{toc}{subsection}{第10章\hspace{0.5cm}基本模板的术语}
	\subfile{content/1/chapter10/0.tex}
	
	\subsubsection*{\zihao{3} 10.1.\hspace{0.2cm}“类模板”还是“模板类”?}
	\addcontentsline{toc}{subsubsection}{10.1.\hspace{0.2cm}“类模板”还是“模板类”?}
	\subfile{content/1/chapter10/1.tex}
	
	\subsubsection*{\zihao{3} 10.2.\hspace{0.2cm}替换、实例化和特化}
	\addcontentsline{toc}{subsubsection}{10.2.\hspace{0.2cm}替换、实例化和特化}
	\subfile{content/1/chapter10/2.tex}
	
	\subsubsection*{\zihao{3} 10.3.\hspace{0.2cm}声明和定义}
	\addcontentsline{toc}{subsubsection}{10.3.\hspace{0.2cm}声明和定义}
	\subfile{content/1/chapter10/3.tex}
	
	\subsubsection*{\zihao{3} 10.4.\hspace{0.2cm}定义规则}
	\addcontentsline{toc}{subsubsection}{10.4.\hspace{0.2cm}定义规则}
	\subfile{content/1/chapter10/4.tex}
	
	\subsubsection*{\zihao{3} 10.5.\hspace{0.2cm}模板实参与模板形参}
	\addcontentsline{toc}{subsubsection}{10.5.\hspace{0.2cm}模板实参与模板形参}
	\subfile{content/1/chapter10/5.tex}
	
	\subsubsection*{\zihao{3} 10.6.\hspace{0.2cm}总结}
	\addcontentsline{toc}{subsubsection}{10.6.\hspace{0.2cm}总结}
	\subfile{content/1/chapter10/6.tex}
	\newpage
	
	\subsection*{\zihao{2} 第11章\hspace{0.5cm}通用库}
	\addcontentsline{toc}{subsection}{第11章\hspace{0.5cm}通用库}
	\subfile{content/1/chapter11/0.tex}
	
	\subsubsection*{\zihao{3} 11.1.\hspace{0.2cm}可调用}
	\addcontentsline{toc}{subsubsection}{11.1.\hspace{0.2cm}可调用}
	\subfile{content/1/chapter11/1.tex}
	
	\subsubsection*{\zihao{3} 11.2.\hspace{0.2cm}实现通用库}
	\addcontentsline{toc}{subsubsection}{11.2.\hspace{0.2cm}实现通用库}
	\subfile{content/1/chapter11/2.tex}
	
	\subsubsection*{\zihao{3} 11.3.\hspace{0.2cm}完美转发临时变量}
	\addcontentsline{toc}{subsubsection}{11.3.\hspace{0.2cm}完美转发临时变量}
	\subfile{content/1/chapter11/3.tex}
	
	\subsubsection*{\zihao{3} 11.4.\hspace{0.2cm}模板形参的引用}
	\addcontentsline{toc}{subsubsection}{11.4.\hspace{0.2cm}模板形参的引用}
	\subfile{content/1/chapter11/4.tex}
	
	\subsubsection*{\zihao{3} 11.5.\hspace{0.2cm}缓式评估}
	\addcontentsline{toc}{subsubsection}{11.5.\hspace{0.2cm}缓式评估}
	\subfile{content/1/chapter11/5.tex}
	
	\subsubsection*{\zihao{3} 11.6.\hspace{0.2cm}编写泛型库时需要考虑的事情}
	\addcontentsline{toc}{subsubsection}{11.6.\hspace{0.2cm}编写泛型库时需要考虑的事情}
	\subfile{content/1/chapter11/6.tex}
	
	\subsubsection*{\zihao{3} 11.7.\hspace{0.2cm}总结}
	\addcontentsline{toc}{subsubsection}{11.7.\hspace{0.2cm}总结}
	\subfile{content/1/chapter11/7.tex}
	\newpage
	
	\color{white}
	\section*{\zihao{1}第二部分:深入了解模板}
	\pagecolor{mygray}
	\addcontentsline{toc}{section}{第二部分:深入了解模板}
	\textbf{\subfile{content/2/Section.tex}}
	\newpage
	\color{black}
	\pagecolor{white}
	
	\subsection*{\zihao{2} 第12章\hspace{0.5cm}模板基础}
	\addcontentsline{toc}{subsection}{第12章\hspace{0.5cm}模板基础}
	\subfile{content/2/chapter12/0.tex}
	
	\subsubsection*{\zihao{3} 12.1.\hspace{0.2cm}参数化的声明}
	\addcontentsline{toc}{subsubsection}{12.1.\hspace{0.2cm}参数化的声明}
	\subfile{content/2/chapter12/1.tex}
	
	\subsubsection*{\zihao{3} 12.2.\hspace{0.2cm}模板形参}
	\addcontentsline{toc}{subsubsection}{12.2.\hspace{0.2cm}模板形参}
	\subfile{content/2/chapter12/2.tex}
	
	\subsubsection*{\zihao{3} 12.3.\hspace{0.2cm}模板实参}
	\addcontentsline{toc}{subsubsection}{12.3.\hspace{0.2cm}模板实参}
	\subfile{content/2/chapter12/3.tex}
	
	\subsubsection*{\zihao{3} 12.4.\hspace{0.2cm}可变参数模板}
	\addcontentsline{toc}{subsubsection}{12.4.\hspace{0.2cm}可变参数模板}
	\subfile{content/2/chapter12/4.tex}
	
	\subsubsection*{\zihao{3} 12.5.\hspace{0.2cm}友元}
	\addcontentsline{toc}{subsubsection}{12.5.\hspace{0.2cm}友元}
	\subfile{content/2/chapter12/5.tex}
	
	\subsubsection*{\zihao{3} 12.6.\hspace{0.2cm}后记}
	\addcontentsline{toc}{subsubsection}{12.6.\hspace{0.2cm}后记}
	\subfile{content/2/chapter12/6.tex}
	\newpage
	
	\subsection*{\zihao{2} 第13章\hspace{0.5cm}模板中的名称}
	\addcontentsline{toc}{subsection}{第13章\hspace{0.5cm}模板中的名称}
	\subfile{content/2/chapter13/0.tex}
	
	\subsubsection*{\zihao{3} 13.1.\hspace{0.2cm}名称的分类}
	\addcontentsline{toc}{subsubsection}{13.1.\hspace{0.2cm}名称的分类}
	\subfile{content/2/chapter13/1.tex}
	
	\subsubsection*{\zihao{3} 13.2.\hspace{0.2cm}查找名称}
	\addcontentsline{toc}{subsubsection}{13.2.\hspace{0.2cm}查找名称}
	\subfile{content/2/chapter13/2.tex}
	
	\subsubsection*{\zihao{3} 13.3.\hspace{0.2cm}解析模板}
	\addcontentsline{toc}{subsubsection}{13.3.\hspace{0.2cm}解析模板}
	\subfile{content/2/chapter13/3.tex}
	
	\subsubsection*{\zihao{3} 13.4.\hspace{0.2cm}派生和类模板}
	\addcontentsline{toc}{subsubsection}{13.4.\hspace{0.2cm}派生和类模板}
	\subfile{content/2/chapter13/4.tex}
	
	\subsubsection*{\zihao{3} 13.5.\hspace{0.2cm}后记}
	\addcontentsline{toc}{subsubsection}{13.5.\hspace{0.2cm}后记}
	\subfile{content/2/chapter13/5.tex}
	\newpage
	
	\subsection*{\zihao{2} 第14章\hspace{0.5cm}实例化}
	\addcontentsline{toc}{subsection}{第14章\hspace{0.5cm}实例化}
	\subfile{content/2/chapter14/0.tex}
	
	\subsubsection*{\zihao{3} 14.1.\hspace{0.2cm}按需实例化}
	\addcontentsline{toc}{subsubsection}{14.1.\hspace{0.2cm}按需实例化}
	\subfile{content/2/chapter14/1.tex}
	
	\subsubsection*{\zihao{3} 14.2.\hspace{0.2cm}延迟实例化}
	\addcontentsline{toc}{subsubsection}{14.2.\hspace{0.2cm}延迟实例化}
	\subfile{content/2/chapter14/2.tex}
	
	\subsubsection*{\zihao{3} 14.3.\hspace{0.2cm}C++实例化模型}
	\addcontentsline{toc}{subsubsection}{14.3.\hspace{0.2cm}C++实例化模型}
	\subfile{content/2/chapter14/3.tex}
	
	\subsubsection*{\zihao{3} 14.4.\hspace{0.2cm}实现方案}
	\addcontentsline{toc}{subsubsection}{14.4.\hspace{0.2cm}实现方案}
	\subfile{content/2/chapter14/4.tex}
	
	\subsubsection*{\zihao{3} 14.5.\hspace{0.2cm}显式实例化}
	\addcontentsline{toc}{subsubsection}{14.5.\hspace{0.2cm}显式实例化}
	\subfile{content/2/chapter14/5.tex}
	
	\subsubsection*{\zihao{3} 14.6.\hspace{0.2cm}编译时if语句}
	\addcontentsline{toc}{subsubsection}{14.6.\hspace{0.2cm}编译时if语句}
	\subfile{content/2/chapter14/6.tex}
	
	\subsubsection*{\zihao{3} 14.7.\hspace{0.2cm}标准库中的显式实例化}
	\addcontentsline{toc}{subsubsection}{14.7.\hspace{0.2cm}标准库中的显式实例化}
	\subfile{content/2/chapter14/7.tex}
	
	\subsubsection*{\zihao{3} 14.8.\hspace{0.2cm}后记}
	\addcontentsline{toc}{subsubsection}{14.8.\hspace{0.2cm}后记}
	\subfile{content/2/chapter14/8.tex}
	\newpage
	
	\subsection*{\zihao{2} 第15章\hspace{0.5cm}Template Argument Deduction}
	\addcontentsline{toc}{subsection}{第15章\hspace{0.5cm}Template Argument Deduction}
	\subfile{content/2/chapter15/0.tex}
	
	\subsubsection*{\zihao{3} 15.1.\hspace{0.2cm}The Deduction Process}
	\addcontentsline{toc}{subsubsection}{15.1.\hspace{0.2cm}The Deduction Process}
	\subfile{content/2/chapter15/1.tex}
	
	\subsubsection*{\zihao{3} 15.2.\hspace{0.2cm}Deduced Contexts}
	\addcontentsline{toc}{subsubsection}{15.2.\hspace{0.2cm}Deduced Contexts}
	\subfile{content/2/chapter15/2.tex}
	
	\subsubsection*{\zihao{3} 15.3.\hspace{0.2cm}Special Deduction Situations}
	\addcontentsline{toc}{subsubsection}{15.3.\hspace{0.2cm}Special Deduction Situations}
	\subfile{content/2/chapter15/3.tex}
	
	\subsubsection*{\zihao{3} 15.4.\hspace{0.2cm}Initializer Lists}
	\addcontentsline{toc}{subsubsection}{15.4.\hspace{0.2cm}Initializer Lists}
	\subfile{content/2/chapter15/4.tex}
	
	\subsubsection*{\zihao{3} 15.5.\hspace{0.2cm}Parameter Packs}
	\addcontentsline{toc}{subsubsection}{15.5.\hspace{0.2cm}Parameter Packs}
	\subfile{content/2/chapter15/5.tex}
	
	\subsubsection*{\zihao{3} 15.6.\hspace{0.2cm}Rvalue References}
	\addcontentsline{toc}{subsubsection}{15.6.\hspace{0.2cm}Rvalue References}
	\subfile{content/2/chapter15/6.tex}
	
	\subsubsection*{\zihao{3} 15.7.\hspace{0.2cm}SFINAE (Substitution Failure Is Not An Error)}
	\addcontentsline{toc}{subsubsection}{15.7.\hspace{0.2cm}SFINAE (Substitution Failure Is Not An Error)}
	\subfile{content/2/chapter15/7.tex}
	
	\subsubsection*{\zihao{3} 15.8.\hspace{0.2cm}Limitations of Deduction}
	\addcontentsline{toc}{subsubsection}{15.8.\hspace{0.2cm}Limitations of Deduction}
	\subfile{content/2/chapter15/8.tex}
	
	\subsubsection*{\zihao{3} 15.9.\hspace{0.2cm}Explicit Function Template Arguments}
	\addcontentsline{toc}{subsubsection}{15.9.\hspace{0.2cm}Explicit Function Template Arguments}
	\subfile{content/2/chapter15/9.tex}
	
	\subsubsection*{\zihao{3} 15.10.\hspace{0.2cm}Deduction from Initializers and Expressions}
	\addcontentsline{toc}{subsubsection}{15.10.\hspace{0.2cm}Deduction from Initializers and Expressions}
	\subfile{content/2/chapter15/10.tex}

	\subsubsection*{\zihao{3} 15.11.\hspace{0.2cm}Alias Templates}
	\addcontentsline{toc}{subsubsection}{15.11.\hspace{0.2cm}Alias Templates}
	\subfile{content/2/chapter15/11.tex}
	
	\subsubsection*{\zihao{3} 15.12.\hspace{0.2cm}Class Template Argument Deduction}
	\addcontentsline{toc}{subsubsection}{15.12.\hspace{0.2cm}Class Template Argument Deduction}
	\subfile{content/2/chapter15/12.tex}
	
	\subsubsection*{\zihao{3} 15.13.\hspace{0.2cm}后记}
	\addcontentsline{toc}{subsubsection}{15.13.\hspace{0.2cm}后记}
	\subfile{content/2/chapter15/13.tex}
	\newpage
	
	\subsection*{\zihao{2} 第16章\hspace{0.5cm}Specialization and Overloading}
	\addcontentsline{toc}{subsection}{第16章\hspace{0.5cm}Specialization and Overloading}
	\subfile{content/2/chapter16/0.tex}
	
	\subsubsection*{\zihao{3} 16.1.\hspace{0.2cm}When “Generic Code” Doesn’t Quite Cut It}
	\addcontentsline{toc}{subsubsection}{16.1.\hspace{0.2cm}When “Generic Code” Doesn’t Quite Cut It}
	\subfile{content/2/chapter16/1.tex}
	
	\subsubsection*{\zihao{3} 16.2.\hspace{0.2cm}Overloading Function Templates}
	\addcontentsline{toc}{subsubsection}{16.2.\hspace{0.2cm}Overloading Function Templates}
	\subfile{content/2/chapter16/2.tex}
	
	\subsubsection*{\zihao{3} 16.3.\hspace{0.2cm}Explicit Specialization}
	\addcontentsline{toc}{subsubsection}{16.3.\hspace{0.2cm}Explicit Specialization}
	\subfile{content/2/chapter16/3.tex}
	
	\subsubsection*{\zihao{3} 16.4.\hspace{0.2cm}Partial Class Template Specialization}
	\addcontentsline{toc}{subsubsection}{16.4.\hspace{0.2cm}Partial Class Template Specialization}
	\subfile{content/2/chapter16/4.tex}
	
	\subsubsection*{\zihao{3} 16.5.\hspace{0.2cm}Partial Variable Template Specialization}
	\addcontentsline{toc}{subsubsection}{16.5.\hspace{0.2cm}Partial Variable Template Specialization}
	\subfile{content/2/chapter16/5.tex}
	
	\subsubsection*{\zihao{3} 16.6.\hspace{0.2cm}后记}
	\addcontentsline{toc}{subsubsection}{16.6.\hspace{0.2cm}后记}
	\subfile{content/2/chapter16/6.tex}
	\newpage
	
	\subsection*{\zihao{2} 第17章\hspace{0.5cm}Future Directions}
	\addcontentsline{toc}{subsection}{第17章\hspace{0.5cm}Future Directions}
	\subfile{content/2/chapter17/0.tex}
	
	\subsubsection*{\zihao{3} 17.1.\hspace{0.2cm}Relaxed typenameRules}
	\addcontentsline{toc}{subsubsection}{17.1.\hspace{0.2cm}Relaxed typenameRules}
	\subfile{content/2/chapter17/1.tex}
	
	\subsubsection*{\zihao{3} 17.2.\hspace{0.2cm}Generalized Nontype Template Parameters}
	\addcontentsline{toc}{subsubsection}{17.2.\hspace{0.2cm}Generalized Nontype Template Parameters}
	\subfile{content/2/chapter17/2.tex}
	
	\subsubsection*{\zihao{3} 17.3.\hspace{0.2cm}Partial Specialization of Function Templates}
	\addcontentsline{toc}{subsubsection}{17.3.\hspace{0.2cm}Partial Specialization of Function Templates}
	\subfile{content/2/chapter17/3.tex}
	
	\subsubsection*{\zihao{3} 17.4.\hspace{0.2cm}Named Template Arguments}
	\addcontentsline{toc}{subsubsection}{17.4.\hspace{0.2cm}Named Template Arguments}
	\subfile{content/2/chapter17/4.tex}
	
	\subsubsection*{\zihao{3} 17.5.\hspace{0.2cm}Overloaded Class Templates}
	\addcontentsline{toc}{subsubsection}{17.5.\hspace{0.2cm}Overloaded Class Templates}
	\subfile{content/2/chapter17/5.tex}
	
	\subsubsection*{\zihao{3} 17.6.\hspace{0.2cm}Deduction for Nonfinal Pack Expansions}
	\addcontentsline{toc}{subsubsection}{17.6.\hspace{0.2cm}Deduction for Nonfinal Pack Expansions}
	\subfile{content/2/chapter17/6.tex}
	
	\subsubsection*{\zihao{3} 17.7.\hspace{0.2cm}Regularization of void}
	\addcontentsline{toc}{subsubsection}{17.7.\hspace{0.2cm}Regularization of void}
	\subfile{content/2/chapter17/7.tex}
	
	\subsubsection*{\zihao{3} 17.8.\hspace{0.2cm}Type Checking for Templates}
	\addcontentsline{toc}{subsubsection}{17.8.\hspace{0.2cm}Type Checking for Templates}
	\subfile{content/2/chapter17/8.tex}
	
	\subsubsection*{\zihao{3} 17.9.\hspace{0.2cm}Reflective Metaprogramming}
	\addcontentsline{toc}{subsubsection}{17.9.\hspace{0.2cm}Reflective Metaprogramming}
	\subfile{content/2/chapter17/9.tex}
	
	\subsubsection*{\zihao{3} 17.10.\hspace{0.2cm}Pack Facilities}
	\addcontentsline{toc}{subsubsection}{17.10.\hspace{0.2cm}Pack Facilities}
	\subfile{content/2/chapter17/10.tex}
	
	\subsubsection*{\zihao{3} 17.11.\hspace{0.2cm}Modules}
	\addcontentsline{toc}{subsubsection}{17.11.\hspace{0.2cm}Modules}
	\subfile{content/2/chapter17/11.tex}
	\newpage
	
	\color{white}
	\section*{\zihao{1}第三部分:Templates and Design}
	\pagecolor{mygray}
	\addcontentsline{toc}{section}{第三部分:Templates and Design}
	\textbf{\subfile{content/3/Section.tex}}
	\newpage
	\color{black}
	\pagecolor{white}
	\newpage
	
	\subsection*{\zihao{2} 第18章\hspace{0.5cm}The Polymorphic Power of Templates}
	\addcontentsline{toc}{subsection}{第18章\hspace{0.5cm}The Polymorphic Power of Templates}
	\subfile{content/3/chapter18/0.tex}
	
	\subsubsection*{\zihao{3} 18.1.\hspace{0.2cm}Dynamic Polymorphism}
	\addcontentsline{toc}{subsubsection}{18.1.\hspace{0.2cm}Dynamic Polymorphism}
	\subfile{content/3/chapter18/1.tex}
	
	\subsubsection*{\zihao{3} 18.2.\hspace{0.2cm}Static Polymorphism}
	\addcontentsline{toc}{subsubsection}{18.2.\hspace{0.2cm}Static Polymorphism}
	\subfile{content/3/chapter18/2.tex}
	
	\subsubsection*{\zihao{3} 18.3.\hspace{0.2cm}Dynamic versus Static Polymorphism}
	\addcontentsline{toc}{subsubsection}{18.3.\hspace{0.2cm}Dynamic versus Static Polymorphism}
	\subfile{content/3/chapter18/3.tex}
	
	\subsubsection*{\zihao{3} 18.4.\hspace{0.2cm}Using Concepts}
	\addcontentsline{toc}{subsubsection}{18.4.\hspace{0.2cm}Using Concepts}
	\subfile{content/3/chapter18/4.tex}
	
	\subsubsection*{\zihao{3} 18.5.\hspace{0.2cm}New Forms of Design Patterns}
	\addcontentsline{toc}{subsubsection}{18.5.\hspace{0.2cm}New Forms of Design Patterns}
	\subfile{content/3/chapter18/5.tex}
	
	\subsubsection*{\zihao{3} 18.6.\hspace{0.2cm}Generic Programming}
	\addcontentsline{toc}{subsubsection}{18.6.\hspace{0.2cm}Generic Programming}
	\subfile{content/3/chapter18/6.tex}
	
	\subsubsection*{\zihao{3} 18.7.\hspace{0.2cm}后记}
	\addcontentsline{toc}{subsubsection}{18.7.\hspace{0.2cm}后记}
	\subfile{content/3/chapter18/7.tex}
	\newpage
	
	\subsection*{\zihao{2} 第19章\hspace{0.5cm}Implementing Traits}
	\addcontentsline{toc}{subsection}{第19章\hspace{0.5cm}Implementing Traits}
	\subfile{content/3/chapter19/0.tex}
	
	\subsubsection*{\zihao{3} 19.1.\hspace{0.2cm}An Example: Accumulating a Sequence}
	\addcontentsline{toc}{subsubsection}{19.1.\hspace{0.2cm}An Example: Accumulating a Sequence}
	\subfile{content/3/chapter19/1.tex}
	
	\subsubsection*{\zihao{3} 19.2.\hspace{0.2cm}Traits versus Policies and Policy Classes}
	\addcontentsline{toc}{subsubsection}{19.2.\hspace{0.2cm}Traits versus Policies and Policy Classes}
	\subfile{content/3/chapter19/2.tex}
	
	\subsubsection*{\zihao{3} 19.3.\hspace{0.2cm}Type Functions}
	\addcontentsline{toc}{subsubsection}{19.3.\hspace{0.2cm}Type Functions}
	\subfile{content/3/chapter19/3.tex}
	
	\subsubsection*{\zihao{3} 19.4.\hspace{0.2cm}SFINAE-Based Traits}
	\addcontentsline{toc}{subsubsection}{19.4.\hspace{0.2cm}SFINAE-Based Traits}
	\subfile{content/3/chapter19/4.tex}
	
	\subsubsection*{\zihao{3} 19.5.\hspace{0.2cm}IsConvertibleT}
	\addcontentsline{toc}{subsubsection}{19.5.\hspace{0.2cm}IsConvertibleT}
	\subfile{content/3/chapter19/5.tex}
	
	\subsubsection*{\zihao{3} 19.6.\hspace{0.2cm}Detecting Members}
	\addcontentsline{toc}{subsubsection}{19.6.\hspace{0.2cm}Detecting Members}
	\subfile{content/3/chapter19/6.tex}
	
	\subsubsection*{\zihao{3} 19.7.\hspace{0.2cm}Other Traits Techniques}
	\addcontentsline{toc}{subsubsection}{19.7.\hspace{0.2cm}Other Traits Techniques}
	\subfile{content/3/chapter19/7.tex}
	
	\subsubsection*{\zihao{3} 19.8.\hspace{0.2cm}Type Classification}
	\addcontentsline{toc}{subsubsection}{19.8.\hspace{0.2cm}Type Classification}
	\subfile{content/3/chapter19/8.tex}
	
	\subsubsection*{\zihao{3} 19.9.\hspace{0.2cm}Policy Traits}
	\addcontentsline{toc}{subsubsection}{19.9.\hspace{0.2cm}Policy Traits}
	\subfile{content/3/chapter19/9.tex}
	
	\subsubsection*{\zihao{3} 19.10.\hspace{0.2cm}In the Standard Library}
	\addcontentsline{toc}{subsubsection}{19.10.\hspace{0.2cm}In the Standard Library}
	\subfile{content/3/chapter19/10.tex}
	
	\subsubsection*{\zihao{3} 19.11.\hspace{0.2cm}后记}
	\addcontentsline{toc}{subsubsection}{19.11.\hspace{0.2cm}后记}
	\subfile{content/3/chapter19/11.tex}
	\newpage
	
	\subsection*{\zihao{2} 第20章\hspace{0.5cm}Overloading on Type Properties}
	\addcontentsline{toc}{subsection}{第20章\hspace{0.5cm}Overloading on Type Properties}
	\subfile{content/3/chapter20/0.tex}
	
	\subsubsection*{\zihao{3} 20.1.\hspace{0.2cm}Algorithm Specialization}
	\addcontentsline{toc}{subsubsection}{20.1.\hspace{0.2cm}Algorithm Specialization}
	\subfile{content/3/chapter20/1.tex}
	
	\subsubsection*{\zihao{3} 20.2.\hspace{0.2cm}Tag Dispatching}
	\addcontentsline{toc}{subsubsection}{20.2.\hspace{0.2cm}Tag Dispatching}
	\subfile{content/3/chapter20/2.tex}
	
	\subsubsection*{\zihao{3} 20.3.\hspace{0.2cm}Enabling/Disabling Function Templates}
	\addcontentsline{toc}{subsubsection}{20.3.\hspace{0.2cm}Enabling/Disabling Function Templates}
	\subfile{content/3/chapter20/3.tex}
	
	\subsubsection*{\zihao{3} 20.4.\hspace{0.2cm}Class Specialization}
	\addcontentsline{toc}{subsubsection}{20.4.\hspace{0.2cm}Class Specialization}
	\subfile{content/3/chapter20/4.tex}
	
	\subsubsection*{\zihao{3} 20.5.\hspace{0.2cm}Instantiation-Safe Templates}
	\addcontentsline{toc}{subsubsection}{20.5.\hspace{0.2cm}Instantiation-Safe Templates}
	\subfile{content/3/chapter20/5.tex}
	
	\subsubsection*{\zihao{3} 20.6.\hspace{0.2cm}In the Standard Library}
	\addcontentsline{toc}{subsubsection}{20.6.\hspace{0.2cm}In the Standard Library}
	\subfile{content/3/chapter20/6.tex}
	
	\subsubsection*{\zihao{3} 20.7.\hspace{0.2cm}后记}
	\addcontentsline{toc}{subsubsection}{20.7.\hspace{0.2cm}后记}
	\subfile{content/3/chapter20/7.tex}
	\newpage
	
	\subsection*{\zihao{2} 第21章\hspace{0.5cm}Templates and Inheritance}
	\addcontentsline{toc}{subsection}{第21章\hspace{0.5cm}Templates and Inheritance}
	\subfile{content/3/chapter21/0.tex}
	
	\subsubsection*{\zihao{3} 21.1.\hspace{0.2cm}The Empty Base Class Optimization (EBCO)}
	\addcontentsline{toc}{subsubsection}{21.1.\hspace{0.2cm}The Empty Base Class Optimization (EBCO)}
	\subfile{content/3/chapter21/1.tex}
	
	\subsubsection*{\zihao{3} 21.2.\hspace{0.2cm}The Curiously Recurring Template Pattern (CRTP)}
	\addcontentsline{toc}{subsubsection}{21.2.\hspace{0.2cm}The Curiously Recurring Template Pattern (CRTP)}
	\subfile{content/3/chapter21/2.tex}
	
	\subsubsection*{\zihao{3} 21.3.\hspace{0.2cm}Mixins}
	\addcontentsline{toc}{subsubsection}{21.3.\hspace{0.2cm}Mixins}
	\subfile{content/3/chapter21/3.tex}
	
	\subsubsection*{\zihao{3} 21.4.\hspace{0.2cm}Named Template Arguments}
	\addcontentsline{toc}{subsubsection}{21.4.\hspace{0.2cm}Named Template Arguments}
	\subfile{content/3/chapter21/4.tex}
	
	\subsubsection*{\zihao{3} 21.5.\hspace{0.2cm}后记}
	\addcontentsline{toc}{subsubsection}{21.5.\hspace{0.2cm}后记}
	\subfile{content/3/chapter21/5.tex}
	\newpage
	
	\subsection*{\zihao{2} 第22章\hspace{0.5cm}Bridging Static and Dynamic Polymorphism}
	\addcontentsline{toc}{subsection}{第22章\hspace{0.5cm}Bridging Static and Dynamic Polymorphism}
	\subfile{content/3/chapter22/0.tex}
	
	\subsubsection*{\zihao{3} 22.1.\hspace{0.2cm}Function Objects, Pointers, and std::function<>}
	\addcontentsline{toc}{subsubsection}{22.1.\hspace{0.2cm}Function Objects, Pointers, and std::function<>}
	\subfile{content/3/chapter22/1.tex}
	
	\subsubsection*{\zihao{3} 22.2.\hspace{0.2cm}Generalized Function Pointers}
	\addcontentsline{toc}{subsubsection}{22.2.\hspace{0.2cm}Generalized Function Pointers}
	\subfile{content/3/chapter22/2.tex}
	
	\subsubsection*{\zihao{3} 22.3.\hspace{0.2cm}Bridge Interface}
	\addcontentsline{toc}{subsubsection}{22.3.\hspace{0.2cm}Bridge Interface}
	\subfile{content/3/chapter22/3.tex}
	
	\subsubsection*{\zihao{3} 22.4.\hspace{0.2cm}Type Erasure}
	\addcontentsline{toc}{subsubsection}{22.4.\hspace{0.2cm}Type Erasure}
	\subfile{content/3/chapter22/4.tex}
	
	\subsubsection*{\zihao{3} 22.5.\hspace{0.2cm}Optional Bridging}
	\addcontentsline{toc}{subsubsection}{22.5.\hspace{0.2cm}Optional Bridging}
	\subfile{content/3/chapter22/5.tex}
	
	\subsubsection*{\zihao{3} 22.6.\hspace{0.2cm}Performance Considerations}
	\addcontentsline{toc}{subsubsection}{22.6.\hspace{0.2cm}Performance Considerations}
	\subfile{content/3/chapter22/6.tex}
	
	\subsubsection*{\zihao{3} 22.7.\hspace{0.2cm}后记}
	\addcontentsline{toc}{subsubsection}{22.7.\hspace{0.2cm}后记}
	\subfile{content/3/chapter22/7.tex}
	\newpage
	
	\subsection*{\zihao{2} 第23章\hspace{0.5cm}Metaprogramming}
	\addcontentsline{toc}{subsection}{第23章\hspace{0.5cm}Metaprogramming}
	\subfile{content/3/chapter23/0.tex}
	
	\subsubsection*{\zihao{3} 23.1.\hspace{0.2cm}The State of Modern C++ Metaprogramming}
	\addcontentsline{toc}{subsubsection}{23.1.\hspace{0.2cm}The State of Modern C++ Metaprogramming}
	\subfile{content/3/chapter23/1.tex}
	
	\subsubsection*{\zihao{3} 23.2.\hspace{0.2cm}The Dimensions of Reflective Metaprogramming}
	\addcontentsline{toc}{subsubsection}{23.2.\hspace{0.2cm}The Dimensions of Reflective Metaprogramming}
	\subfile{content/3/chapter23/2.tex}
	
	\subsubsection*{\zihao{3} 23.3.\hspace{0.2cm}The Cost of Recursive Instantiation}
	\addcontentsline{toc}{subsubsection}{23.3.\hspace{0.2cm}The Cost of Recursive Instantiation}
	\subfile{content/3/chapter23/3.tex}
	
	\subsubsection*{\zihao{3} 23.4.\hspace{0.2cm}Computational Completeness}
	\addcontentsline{toc}{subsubsection}{23.4.\hspace{0.2cm}Computational Completeness}
	\subfile{content/3/chapter23/4.tex}
	
	\subsubsection*{\zihao{3} 23.5.\hspace{0.2cm}Recursive Instantiation versus Recursive Template Arguments}
	\addcontentsline{toc}{subsubsection}{23.5.\hspace{0.2cm}Recursive Instantiation versus Recursive Template Arguments}
	\subfile{content/3/chapter23/5.tex}
	
	\subsubsection*{\zihao{3} 23.6.\hspace{0.2cm}Enumeration Values versus Static Constants}
	\addcontentsline{toc}{subsubsection}{23.6.\hspace{0.2cm}Enumeration Values versus Static Constants}
	\subfile{content/3/chapter23/6.tex}
	
	\subsubsection*{\zihao{3} 23.7.\hspace{0.2cm}后记}
	\addcontentsline{toc}{subsubsection}{23.7.\hspace{0.2cm}后记}
	\subfile{content/3/chapter23/7.tex}
	\newpage
	
	\subsection*{\zihao{2} 第24章\hspace{0.5cm}Typelists}
	\addcontentsline{toc}{subsection}{第24章\hspace{0.5cm}Typelists}
	\subfile{content/3/chapter24/0.tex}
	
	\subsubsection*{\zihao{3} 24.1.\hspace{0.2cm}Anatomy of a Typelist}
	\addcontentsline{toc}{subsubsection}{24.1.\hspace{0.2cm}Anatomy of a Typelist}
	\subfile{content/3/chapter24/1.tex}
	
	\subsubsection*{\zihao{3} 24.2.\hspace{0.2cm}Typelist Algorithms}
	\addcontentsline{toc}{subsubsection}{24.2.\hspace{0.2cm}Typelist Algorithms}
	\subfile{content/3/chapter24/2.tex}
	
	\subsubsection*{\zihao{3} 24.3.\hspace{0.2cm}Nontype Typelists}
	\addcontentsline{toc}{subsubsection}{24.3.\hspace{0.2cm}Nontype Typelists}
	\subfile{content/3/chapter24/3.tex}
	
	\subsubsection*{\zihao{3} 24.4.\hspace{0.2cm}Optimizing Algorithms with Pack Expansions}
	\addcontentsline{toc}{subsubsection}{24.4.\hspace{0.2cm}Optimizing Algorithms with Pack Expansions}
	\subfile{content/3/chapter24/4.tex}
	
	\subsubsection*{\zihao{3} 24.5.\hspace{0.2cm}Cons-style Typelists}
	\addcontentsline{toc}{subsubsection}{24.5.\hspace{0.2cm}Cons-style Typelists}
	\subfile{content/3/chapter24/5.tex}
	
	\subsubsection*{\zihao{3} 24.6.\hspace{0.2cm}后记}
	\addcontentsline{toc}{subsubsection}{24.6.\hspace{0.2cm}后记}
	\subfile{content/3/chapter24/6.tex}
	\newpage
	
	\subsection*{\zihao{2} 第25章\hspace{0.5cm}Tuples}
	\addcontentsline{toc}{subsection}{第25章\hspace{0.5cm}Tuples}
	\subfile{content/3/chapter25/0.tex}
	
	\subsubsection*{\zihao{3} 25.1.\hspace{0.2cm}Basic Tuple Design}
	\addcontentsline{toc}{subsubsection}{25.1.\hspace{0.2cm}Basic Tuple Design}
	\subfile{content/3/chapter25/1.tex}
	
	\subsubsection*{\zihao{3} 25.2.\hspace{0.2cm}Basic Tuple Operations}
	\addcontentsline{toc}{subsubsection}{25.2.\hspace{0.2cm}Basic Tuple Operations}
	\subfile{content/3/chapter25/2.tex}
	
	\subsubsection*{\zihao{3} 25.3.\hspace{0.2cm}Tuple Algorithms}
	\addcontentsline{toc}{subsubsection}{25.3.\hspace{0.2cm}Tuple Algorithms}
	\subfile{content/3/chapter25/3.tex}
	
	\subsubsection*{\zihao{3} 25.4.\hspace{0.2cm}Expanding Tuples}
	\addcontentsline{toc}{subsubsection}{25.4.\hspace{0.2cm}Expanding Tuples}
	\subfile{content/3/chapter25/4.tex}
	
	\subsubsection*{\zihao{3} 25.5.\hspace{0.2cm}Optimizing Tuple}
	\addcontentsline{toc}{subsubsection}{25.5.\hspace{0.2cm}Optimizing Tuple}
	\subfile{content/3/chapter25/5.tex}
	
	\subsubsection*{\zihao{3} 25.6.\hspace{0.2cm}Tuple Subscript}
	\addcontentsline{toc}{subsubsection}{25.6.\hspace{0.2cm}Tuple Subscript}
	\subfile{content/3/chapter25/6.tex}
	
	\subsubsection*{\zihao{3} 25.7.\hspace{0.2cm}后记}
	\addcontentsline{toc}{subsubsection}{25.7.\hspace{0.2cm}后记}
	\subfile{content/3/chapter25/7.tex}
	\newpage
	
	\subsection*{\zihao{2} 第26章\hspace{0.5cm}Discriminated Unions}
	\addcontentsline{toc}{subsection}{第26章\hspace{0.5cm}Discriminated Unions}
	\subfile{content/3/chapter26/0.tex}
	
	\subsubsection*{\zihao{3} 26.1.\hspace{0.2cm}Storage}
	\addcontentsline{toc}{subsubsection}{26.1.\hspace{0.2cm}Storage}
	\subfile{content/3/chapter26/1.tex}
	
	\subsubsection*{\zihao{3} 26.2.\hspace{0.2cm}Design}
	\addcontentsline{toc}{subsubsection}{26.2.\hspace{0.2cm}Design}
	\subfile{content/3/chapter26/2.tex}
	
	\subsubsection*{\zihao{3} 26.3.\hspace{0.2cm}Value Query and Extraction}
	\addcontentsline{toc}{subsubsection}{26.3.\hspace{0.2cm}Value Query and Extraction}
	\subfile{content/3/chapter26/3.tex}
	
	\subsubsection*{\zihao{3} 26.4.\hspace{0.2cm}Element Initialization, Assignment and Destruction}
	\addcontentsline{toc}{subsubsection}{26.4.\hspace{0.2cm}Element Initialization, Assignment and Destruction}
	\subfile{content/3/chapter26/4.tex}
	
	\subsubsection*{\zihao{3} 26.5.\hspace{0.2cm}Visitors}
	\addcontentsline{toc}{subsubsection}{26.5.\hspace{0.2cm}Visitors}
	\subfile{content/3/chapter26/5.tex}
	
	\subsubsection*{\zihao{3} 26.6.\hspace{0.2cm}Variant Initialization and Assignment}
	\addcontentsline{toc}{subsubsection}{26.6.\hspace{0.2cm}Variant Initialization and Assignment}
	\subfile{content/3/chapter26/6.tex}
	
	\subsubsection*{\zihao{3} 26.7.\hspace{0.2cm}后记}
	\addcontentsline{toc}{subsubsection}{26.7.\hspace{0.2cm}后记}
	\subfile{content/3/chapter26/7.tex}
	\newpage
	
	\subsection*{\zihao{2} 第27章\hspace{0.5cm}Expression Templates}
	\addcontentsline{toc}{subsection}{第27章\hspace{0.5cm}Expression Templates}
	\subfile{content/3/chapter27/0.tex}
	
	\subsubsection*{\zihao{3} 27.1.\hspace{0.2cm}Temporaries and Split Loops}
	\addcontentsline{toc}{subsubsection}{27.1.\hspace{0.2cm}Temporaries and Split Loops}
	\subfile{content/3/chapter27/1.tex}
	
	\subsubsection*{\zihao{3} 27.2.\hspace{0.2cm}Encoding Expressions in Template Arguments}
	\addcontentsline{toc}{subsubsection}{27.2.\hspace{0.2cm}Encoding Expressions in Template Arguments}
	\subfile{content/3/chapter27/2.tex}
	
	\subsubsection*{\zihao{3} 27.3.\hspace{0.2cm}Performance and Limitations of Expression Templates}
	\addcontentsline{toc}{subsubsection}{27.3.\hspace{0.2cm}Performance and Limitations of Expression Templates}
	\subfile{content/3/chapter27/3.tex}
	
	\subsubsection*{\zihao{3} 27.4.\hspace{0.2cm}后记}
	\addcontentsline{toc}{subsubsection}{27.4.\hspace{0.2cm}后记}
	\subfile{content/3/chapter27/4.tex}
	\newpage
	
	\subsection*{\zihao{2} 第28章\hspace{0.5cm}Discriminated Unions}
	\addcontentsline{toc}{subsection}{第28章\hspace{0.5cm}Discriminated Unions}
	\subfile{content/3/chapter28/0.tex}
	
	\subsubsection*{\zihao{3} 28.1.\hspace{0.2cm}Storage}
	\addcontentsline{toc}{subsubsection}{28.1.\hspace{0.2cm}Storage}
	\subfile{content/3/chapter28/1.tex}
	
	\subsubsection*{\zihao{3} 28.2.\hspace{0.2cm}Design}
	\addcontentsline{toc}{subsubsection}{28.2.\hspace{0.2cm}Design}
	\subfile{content/3/chapter28/2.tex}
	
	\subsubsection*{\zihao{3} 28.3.\hspace{0.2cm}Value Query and Extraction}
	\addcontentsline{toc}{subsubsection}{28.3.\hspace{0.2cm}Value Query and Extraction}
	\subfile{content/3/chapter28/3.tex}
	
	\subsubsection*{\zihao{3} 28.4.\hspace{0.2cm}Element Initialization, Assignment and Destruction}
	\addcontentsline{toc}{subsubsection}{28.4.\hspace{0.2cm}Element Initialization, Assignment and Destruction}
	\subfile{content/3/chapter28/4.tex}
	
	\subsubsection*{\zihao{3} 28.5.\hspace{0.2cm}Visitors}
	\addcontentsline{toc}{subsubsection}{28.5.\hspace{0.2cm}Visitors}
	\subfile{content/3/chapter28/5.tex}
	
	\subsubsection*{\zihao{3} 28.6.\hspace{0.2cm}后记}
	\addcontentsline{toc}{subsubsection}{28.6.\hspace{0.2cm}后记}
	\subfile{content/3/chapter28/6.tex}
	\newpage
	
	\section*{\zihao{2} 附录A:The One-Definition Rule}
	\addcontentsline{toc}{section}{附录A:The One-Definition Rule}
	\subfile{content/Appendix/A/0.tex}
	
	\subsubsection*{\zihao{3} A.1\hspace{0.2cm}Translation Units}
	\addcontentsline{toc}{subsubsection}{A.1\hspace{0.2cm}Translation Units}
	\subfile{content/Appendix/A/1.tex}
	
	\subsubsection*{\zihao{3} A.2\hspace{0.2cm}Declarations and Definitions}
	\addcontentsline{toc}{subsubsection}{A.2\hspace{0.2cm}Declarations and Definitions}
	\subfile{content/Appendix/A/2.tex}
	
	\subsubsection*{\zihao{3} A.3\hspace{0.2cm}The One-Definition Rule in Detail}
	\addcontentsline{toc}{subsubsection}{A.3\hspace{0.2cm}The One-Definition Rule in Detail}
	\subfile{content/Appendix/A/3.tex}
	\newpage
	
	\section*{\zihao{2} 附录B:Value Categories}
	\addcontentsline{toc}{section}{附录B:Value Categories}
	\subfile{content/Appendix/B/0.tex}
	
	\subsubsection*{\zihao{3} B.1\hspace{0.2cm}Traditional Lvalues and Rvalues}
	\addcontentsline{toc}{subsubsection}{B.1\hspace{0.2cm}Traditional Lvalues and Rvalues}
	\subfile{content/Appendix/B/1.tex}
	
	\subsubsection*{\zihao{3} B.2\hspace{0.2cm}Value Categories Since C++11}
	\addcontentsline{toc}{subsubsection}{B.2\hspace{0.2cm}Value Categories Since C++11}
	\subfile{content/Appendix/B/2.tex}
	
	\subsubsection*{\zihao{3} B.3\hspace{0.2cm}Checking Value Categories with decltype}
	\addcontentsline{toc}{subsubsection}{B.3\hspace{0.2cm}Checking Value Categories with decltype}
	\subfile{content/Appendix/B/3.tex}
	
	\subsubsection*{\zihao{3} B.4\hspace{0.2cm}Reference Types}
	\addcontentsline{toc}{subsubsection}{B.4\hspace{0.2cm}Reference Types}
	\subfile{content/Appendix/B/4.tex}
	\newpage
	
	\section*{\zihao{2} 附录C:Overload Resolution}
	\addcontentsline{toc}{section}{附录C:Overload Resolution}
	\subfile{content/Appendix/C/0.tex}
	
	\subsubsection*{\zihao{3} C.1\hspace{0.2cm}When Does Overload Resolution Kick In?}
	\addcontentsline{toc}{subsubsection}{C.1\hspace{0.2cm}When Does Overload Resolution Kick In?}
	\subfile{content/Appendix/C/1.tex}
	
	\subsubsection*{\zihao{3} C.2\hspace{0.2cm}Simplified Overload Resolution}
	\addcontentsline{toc}{subsubsection}{C.2\hspace{0.2cm}Simplified Overload Resolution}
	\subfile{content/Appendix/C/2.tex}
	
	\subsubsection*{\zihao{3} C.3\hspace{0.2cm}Overloading Details}
	\addcontentsline{toc}{subsubsection}{C.3\hspace{0.2cm}Overloading Details}
	\subfile{content/Appendix/C/3.tex}
	\newpage
	
	\section*{\zihao{2} 附录D:Standard Type Utilities}
	\addcontentsline{toc}{section}{附录D:Standard Type Utilities}
	\subfile{content/Appendix/D/0.tex}
	
	\subsubsection*{\zihao{3} D.1\hspace{0.2cm}Using Type Traits}
	\addcontentsline{toc}{subsubsection}{D.1\hspace{0.2cm}Using Type Traits}
	\subfile{content/Appendix/D/1.tex}
	
	\subsubsection*{\zihao{3} D.2\hspace{0.2cm}cmPrimary and Composite Type Categories}
	\addcontentsline{toc}{subsubsection}{D.2\hspace{0.2cm}Primary and Composite Type Categories}
	\subfile{content/Appendix/D/2.tex}
	
	\subsubsection*{\zihao{3} D.3\hspace{0.2cm}Type Properties and Operations}
	\addcontentsline{toc}{subsubsection}{D.3\hspace{0.2cm}Type Properties and Operations}
	\subfile{content/Appendix/D/3.tex}
	
	\subsubsection*{\zihao{3} D.4\hspace{0.2cm}Type Construction}
	\addcontentsline{toc}{subsubsection}{D.4\hspace{0.2cm}Type Construction}
	\subfile{content/Appendix/D/4.tex}
	
	\subsubsection*{\zihao{3} D.5\hspace{0.2cm}Other Traits}
	\addcontentsline{toc}{subsubsection}{D.5\hspace{0.2cm}Other Traits}
	\subfile{content/Appendix/D/5.tex}
	
	\subsubsection*{\zihao{3} D.6\hspace{0.2cm}Combining Type Traits}
	\addcontentsline{toc}{subsubsection}{D.6\hspace{0.2cm}Combining Type Traits}
	\subfile{content/Appendix/D/6.tex}
	
	\subsubsection*{\zihao{3} D.7\hspace{0.2cm}Other Utilitiess}
	\addcontentsline{toc}{subsubsection}{D.7\hspace{0.2cm}Other Utilities}
	\subfile{content/Appendix/D/7.tex}
	\newpage
	
	\section*{\zihao{2} 附录E:Concepts}
	\addcontentsline{toc}{section}{附录E:Concepts}
	\subfile{content/Appendix/E/0.tex}
	
	\subsubsection*{\zihao{3} E.1\hspace{0.2cm}Using Concepts}
	\addcontentsline{toc}{subsubsection}{E.1\hspace{0.2cm}Using Concepts}
	\subfile{content/Appendix/E/1.tex}
	
	\subsubsection*{\zihao{3} E.2\hspace{0.2cm}Defining Concepts}
	\addcontentsline{toc}{subsubsection}{E.2\hspace{0.2cm}Defining Concepts}
	\subfile{content/Appendix/E/2.tex}
	
	\subsubsection*{\zihao{3} E.3\hspace{0.2cm}Overloading on Constraints}
	\addcontentsline{toc}{subsubsection}{E.3\hspace{0.2cm}Overloading on Constraints}
	\subfile{content/Appendix/E/3.tex}
	
	\subsubsection*{\zihao{3} E.4\hspace{0.2cm}Concept Tips}
	\addcontentsline{toc}{subsubsection}{E.4\hspace{0.2cm}Concept Tips}
	\subfile{content/Appendix/E/4.tex}
	\newpage

	\section*{\zihao{2} Bibliography}
	\addcontentsline{toc}{section}{Bibliography}
	\subfile{content/Bibliography.tex}
	\newpage
	
	\section*{\zihao{2} Glossary}
	\addcontentsline{toc}{section}{Glossary}
	\subfile{content/Glossary.tex}

\end{sloppypar}
\end{document}




函数模板代表一组函数,提供了适用于不同数据类型的行为。除了某些信息未明确指定外,看起来就和普通函数一样。这些未指定的信息就是参数化信息。

\subsubsubsection{1.1.1\hspace{0.2cm}定义模板}

下面就是一个函数模板,返回两个数的最大值:

\noindent
\textit{basics/max1.hpp}
\begin{lstlisting}[style=styleCXX]
template<typename T>
T max (T a, T b)
{
	// if b < a then yield a else yield b
	return b < a ? a : b;
}
\end{lstlisting}

这个模板定义指定了一个函数组,返回两个值的最大值,a和b作为函数的参数。

\begin{tcolorbox}[colback=webgreen!5!white,colframe=webgreen!75!black]
\hspace*{0.75cm}根据[StepanovNotes],max()模板有意地返回b < a ? a: b,而不是a < b ? b: a,是为了确保函数行为的正确性。
\end{tcolorbox}

参数类型为模板参数T,模板参数必须使用以下形式的进行声明:

\begin{lstlisting}[style=styleCXX]
template<逗号分割的模板参数>
\end{lstlisting}

示例中,参数列表是typename T。注意<和>尖括号。关键字typename引入了一个类型参数。这是C++程序中最常见的模板参数类型,也可以使用其他参数,会在后面进行讨论(参见第3章)。

类型参数是T,也可以使用其他标识符作为参数名(T是惯例罢了)。类型参数表示在调用函数时才确定的类型,可以使用支持模板使用操作的类型(基本类型、类等)。因为a和b使用小于操作符进行比较,所以类型T必须支持小于操作符。max()的定义可能不太明显,类型T的值必须可复制,以便返回。

\begin{tcolorbox}[colback=webgreen!5!white,colframe=webgreen!75!black]
\hspace*{0.75cm}C++17之前,类型T也必须可复制才能传递参数。C++17以后,即使复制构造函数和移动构造函数都无效,也可以传递临时变量(右值,参见附录B)。
\end{tcolorbox}

由于一些历史原因,还可以使用关键字class来定义类型参数。关键字typename在C++98标准中出现得较晚,所以在此之前,关键字class是引入类型参数的唯一方法,现在这种方法仍然有效。因此,模板max()可以等价地定义为:

\begin{lstlisting}[style=styleCXX]
template<class T>
T max (T a, T b)
{
	return b < a ? a : b;
}
\end{lstlisting}

即使使用class,模板参数也可以使用任意类型。但这样使用class可能会引起误会(不仅是类可以替换为T),所以更推荐使用typename。但与类声明不同,在声明类型参数时,不能使用关键字struct来代替typename。

\subsubsubsection{1.1.2\hspace{0.2cm}使用模板}

下面的展示了如何使用函数模板max():

\hspace*{\fill} \\ %插入空行
\noindent
\textit{basics/max1.cpp}
\begin{lstlisting}[style=styleCXX]
#include "max1.hpp"
#include <iostream>
#include <string>

int main()
{
	int i = 42;
	std::cout << "max(7,i): " << ::max(7,i) << '\n';
	
	double f1 = 3.4;
	double f2 = -6.7;
	std::cout << "max(f1,f2): " << ::max(f1,f2) << '\n';
	
	std::string s1 = "mathematics";
	std::string s2 = "math";
	std::cout << "max(s1,s2): " << ::max(s1,s2) << '\n';
}
\end{lstlisting}

这里,max()调用了三次:int,double,std::string。输出如下:

\begin{tcblisting}{commandshell={}}
max(7,i): 42
max(f1,f2): 3.4
max(s1,s2): mathematics
\end{tcblisting}

代码中,对max()的使用都用\textbf{::}限定。这是为了确保在全局命名空间中找到max()模板。因为标准库中有一个std::max(),在某些情况下可以使用,但这里可能会产生歧义。

\begin{tcolorbox}[colback=webgreen!5!white,colframe=webgreen!75!black]
\hspace*{0.75cm}命名空间std中定义了一个参数类型(例如std::string),根据C++的查找规则,std中的max()模板函数和全局模板函数都会找到(参见附录C)。
\end{tcolorbox}

模板不会编译成处理任意类型的实体。相反,对于模板所使用的每个类型,会根据模板中生成不同的实体。

\begin{tcolorbox}[colback=webgreen!5!white,colframe=webgreen!75!black]
\hspace*{0.75cm}“一体多用”的方案可以想想,但不会在实践中使用(运行时效率较低)。所有语言规则都基于这样一个原则:对于不同的模板参数,会生成不同的实体。
\end{tcolorbox}

因此,max()对这三种类型进行了分别编译。例如,max()的第一次调用

\begin{lstlisting}[style=styleCXX]
int i = 42;
... max(7,i) ...
\end{lstlisting}

使用int作为模板形参T的函数模板。因此,等同于调用的如下函数:

\begin{lstlisting}[style=styleCXX]
int max (int a, int b)
{
	return b < a ? a : b;
}
\end{lstlisting}

用具体类型替换模板参数的过程称为实例化,会产生一个模板实例。

\begin{tcolorbox}[colback=webgreen!5!white,colframe=webgreen!75!black]
\hspace*{0.75cm}面向对象编程中,术语实例(instance)和实例化(instantiate)用于不同的上下文——类的具体对象。本书针对模板,仅用这两个术语表示对模板的“使用”,除非另有说明。
\end{tcolorbox}

注意,使用函数模板就可以进行实例化,开发者无需单独实例化。

类似地,max()的其他调用实例化了double和std::string的max模板:

\begin{lstlisting}[style=styleCXX]
double max (double, double);
std::string max (std::string, std::string);
\end{lstlisting}

如果结果代码有效,void也是有效的模板参数。例如:

\begin{lstlisting}[style=styleCXX]
template<typename T>
T foo(T*)
{
}

void* vp = nullptr;
foo(vp); // OK: deduces void foo(void*)
\end{lstlisting}

\subsubsubsection{1.1.3\hspace{0.2cm}两阶段翻译}

若为不支持使用操作的类型实例化模板,将导致编译时错误。例如:

\begin{lstlisting}[style=styleCXX]
std::complex<float> c1, c2; // doesn’t provide operator <
...
::max(c1,c2); // ERROR at compile time
\end{lstlisting}

因此,模板“编译”分为两个阶段:

\begin{enumerate}
\item 
若在定义时不进行实例化,则会检查模板代码的正确性,而忽略模板参数。这包括:
\begin{itemize}
\item[-] 
现语法错误,比如:缺少分号。

\item[-]
使用未知名称的模板参数(类型名、函数名……)。

\item[-]
检查(不依赖于模板参数)静态断言。
\end{itemize}

\item 
实例化时,再次检查模板代码,确保生成的代码有效。特别是所有依赖于模板参数的部分,都会进行重复检查。
\end{enumerate}

例如:

\begin{lstlisting}[style=styleCXX]
template<typename T>
void foo(T t)
{
	undeclared(); // 若undeclared()未知,则在第一阶段编译时报错
	undeclared(t); // 若undeclared(T)未知,则在第二阶段编译时报错
	static_assert(sizeof(int) > 10, // 若sizeof(int)<=10,始终断言失败
				  "int too small");
	static_assert(sizeof(T) > 10, // 若示例T的大小小于等于10,则断言失败
	              "T too small");
}
\end{lstlisting}

检查两次名称的情况称为两阶段查找,在第14.3.1节会进行详细讨论。

注意,有些编译器不在第一阶段编译时进行完全的检查。

\begin{tcolorbox}[colback=webgreen!5!white,colframe=webgreen!75!black]
\hspace*{0.75cm}例如,一些版本的Visual C++编译器(如Visual Studio 2013和2015)允许模板参数使用未声明名称,甚至允许一些语法缺陷(如缺少分号)。
\end{tcolorbox}

因此,在模板代码第一次实例化前,可能不会看到已存在的问题。

\hspace*{\fill} \\ %插入空行
\noindent
\textbf{编译和连接}

两阶段翻译在实际处理模板时,会有一个问题:当以数模板实例化的方式使用函数模板时,编译器(在某些时候)需要查看该模板的定义。当函数的声明可以编译,并连接通过时,就打破了普通函数的编译和链接方式。第9章讨论了处理这个问题的方法。现在,我们使用最简单的方法:仅在头文件中实现每个模板。











函数模板提供了适用于不同数据类型的函数行为,函数模板代表的是一组函数。除了某些信息未被明确指定之外,看起来很像普通函数。这些未指定的信息就是参数化的信息。

\subsubsubsection{1.1.1\hspace{0.2cm}Defining the Template}

以下就是一个函数模板,它返回两个数之中的最大值:

\noindent
\textit{asics/max1.hpp}
\begin{lstlisting}[style=styleCXX]
template<typename T>
T max (T a, T b)
{
	// if b < a then yield a else yield b
	return b < a ? a : b;
}
\end{lstlisting}

This template definition specifies a family of functions that return the maximum of two values, which are passed as function parameters a and b.

\begin{tcolorbox}[colback=webgreen!5!white,colframe=webgreen!75!black]
\hspace*{0.75cm}Note that the max() template according to [StepanovNotes] intentionally returns “b < a ? a : b” instead of “a < b ? b : a” to ensure that the function behaves correctly even if the two values are equivalent but not equal.
\end{tcolorbox}

The type of these parameters is left open as template parameter T. As seen in this example, template parameters must be announced with syntax of the following form:

\begin{lstlisting}[style=styleCXX]
template<逗号分割的模板参数>
\end{lstlisting}

\begin{lstlisting}[style=styleCXX]
template<class T>
T max (T a, T b)
{
	return b < a ? a : b;
}
\end{lstlisting}

\subsubsubsection{1.1.2\hspace{0.2cm}Using the Template}

\subsubsubsection{1.1.3\hspace{0.2cm}Two-Phase Translation}
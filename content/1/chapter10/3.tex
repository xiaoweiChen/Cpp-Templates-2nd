
目前,“声明”和“定义”这两个词在本书中只出现过几次。然而,这些词在标准C++中有相当精确的含义。

声明是一种C++构造,在C++作用域中引入或重新引入一个名称。此介绍会包含该名称的部分类别,但进行有效声明时不需要详细信息。例如:

\begin{lstlisting}[style=styleCXX]
class C; // a declaration of C as a class
void f(int p); // a declaration of f() as a function and p as a named parameter
extern int v; // a declaration of v as a variable
\end{lstlisting}

注意,宏定义和goto标签在C++中不是声明。

当声明的结构已知时,或对于变量,必须分配存储空间时,声明就变成了定义。对于类类型定义,必须提供带括号的主体。对于函数定义,这就必须提供(一般情况下)用大括号括起来的函数体,或者必须将函数指定为=default或=delete。对于变量,初始化或缺少extern说明符会导致声明变成定义。下面是补充上述非定义声明的示例:

\begin{tcolorbox}[colback=webgreen!5!white,colframe=webgreen!75!black]
\hspace*{0.75cm}默认函数是由编译器提供默认实现的特殊成员函数,例如:默认复制构造函数。
\end{tcolorbox}

\begin{lstlisting}[style=styleCXX]
class C {}; // definition (and declaration) of class C

void f(int p) { // definition (and declaration) of function f()
	std::cout << p << ’\n’;
}

extern int v = 1; // an initializer makes this a definition for v

int w; // global variable declarations not preceded by
// extern are also definitions
\end{lstlisting}

通过扩展,类模板或函数模板的声明若有主体,就称为定义。因此,

\begin{lstlisting}[style=styleCXX]
template<typename T>
void func (T);
\end{lstlisting}

声明不是定义,

\begin{lstlisting}[style=styleCXX]
template<typename T>
class S {};
\end{lstlisting}

实际上是一个定义。

\subsubsubsection{10.3.1\hspace{0.2cm}完整类型与不完整类型}

类型可以完整的或不完整的,这是一个与声明和定义之间的区别密切相关的概念。有些语言构造需要完整的类型,而其他语言构造也可以使用不完整的类型。

不完整类型是以下类型之一:

\begin{itemize}
\item 
已声明但尚未定义的类类型。

\item 
未指定边界的数组类型。

\item 
不完整类型的数组类型。

\item 
无效类型

\item 
枚举类型,基础类型或枚举值未定义。

\item 
上面应用const和/或volatile的类型。
\end{itemize}

其他类型都是完整的。例如:

\begin{lstlisting}[style=styleCXX]
class C; // C is an incomplete type
C const* cp; // cp is a pointer to an incomplete type
extern C elems[10]; // elems has an incomplete type
extern int arr[]; // arr has an incomplete type
...
class C { }; // C now is a complete type (and therefore cpand elems
			 // no longer refer to an incomplete type)
int arr[10]; // arr now has a complete type
\end{lstlisting}

有关如何处理模板中不完整类型的提示,请参阅11.5节。
























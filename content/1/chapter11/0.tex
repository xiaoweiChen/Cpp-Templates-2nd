So far, our discussion of templates has focused on their specific features, capabilities, and constraints, with immediate tasks and applications in mind (the kind of things we run into as application programmers). However, templates are most effective when used to write generic libraries and frameworks, where our designs have to consider potential uses that are a priori broadly unconstrained. While just about all the content in this book can be applicable to such designs, here are some general issues you should consider when writing portable components that you intend to be usable for as-yet unimagined types.

The list of issues raised here is not complete in any sense, but it summarizes some of the features introduced so far, introduces some additional features, and refers to some features covered later in this book. We hope it will also be a great motivator to read through the many chapters that follow.
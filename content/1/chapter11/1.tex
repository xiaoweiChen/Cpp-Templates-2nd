
许多库包括外部代码向其传递接口。示例包括必须在另一个线程上调度的操作、如何将哈希值存储在哈希表中的函数、对集合中的元素排序的顺序的对象,以及提供默认参数值的通用包装器。标准库在这里也不例外,定义了许多使用此类的组件。

这种情况下使用的术语是回调,作为函数调用参数传递的实体(与模板参数相反),我们保持了这个传统。例如,sort函数可以包含一个回调参数作为“排序条件”,调用该参数来确定一个元素是否在所需排序顺序的另一个元素之前。

C++中,有几种类型可以很好地用于回调,它们既可以作为函数调用参数传递,也可以以f(…)方式直接使用:

\begin{itemize}
\item
函数指针类型

\item
具有重载operator()(函数操作符,有时称为函子)的类类型,包括Lambda

\item
使用转换函数生成指向函数的指针或指向函数引用的类类型
\end{itemize}

这些类型统称为函数对象类型,这种类型的值就是函数对象。

C++标准库引入了更宽泛的可调用类型概念,可以是函数对象类型或成员指针。可调用类型的对象是可调用对象,这里将其称为可调用对象。

泛型代码通常是能够接受任何类型的可调用代码。

\subsubsubsection{11.1.1\hspace{0.2cm}函数对象}

来看看标准库的for\_each()算法是如何实现的(使用我们的“foreach”以避免名称冲突):

\hspace*{\fill} \\ %插入空行
\noindent
\textit{basics/foreach.hpp}
\begin{lstlisting}[style=styleCXX]
template<typename Iter, typename Callable>
void foreach (Iter current, Iter end, Callable op)
{
	while (current != end) { // as long as not reached the end
		op(*current); // call passed operator for current element
		++current; // and move iterator to next element
	}
}
\end{lstlisting}

下面的程序演示了该模板与各种函数对象的使用:

\hspace*{\fill} \\ %插入空行
\noindent
\textit{basics/foreach.cpp}
\begin{lstlisting}[style=styleCXX]
#include <iostream>
#include <vector>
#include "foreach.hpp"
// a function to call:
void func(int i)
{
	std::cout << "func() called for: " << i << ’\n’;
}

// a function object type (for objects that can be used as functions):
class FuncObj {
	public:
	void operator() (int i) const { // Note: const member function
		std::cout << "FuncObj::op() called for: " << i << ’\n’;
	}
};

int main()
{
	std::vector<int> primes = { 2, 3, 5, 7, 11, 13, 17, 19 };
	foreach(primes.begin(), primes.end(), // range
			func); // function as callable (decays to pointer)

	foreach(primes.begin(), primes.end(), // range
			&func); // function pointer as callable

	foreach(primes.begin(), primes.end(), // range
			FuncObj()); // function object as callable

	foreach(primes.begin(), primes.end(), // range
			[] (int i) { // lambda as callable
				std::cout << "lambda called for: " << i << ’\n’;
			});
}
\end{lstlisting}

来仔细看看每种情况:

\begin{itemize}
\item
将函数名作为函数参数传递时,实际上传递的不是函数本身,而是指向函数的指针或引用。与数组一样(参见7.4节),函数实参在按值传递时衰变为指针,对于模板参数类型,将推导出指向函数的指针类型。

就像数组一样,函数可以通过引用传递,但函数类型不能用const限定。若用可调用的const\&类型声明foreach()的最后一个参数,那么const将会忽略。(主流C++编码中很少使用函数引用。)

\item
第二个调用通过传递函数名的地址显式接受函数指针。这相当于第一次调用(其中函数名隐式衰变为指针值),但可能更清晰一些

\item
传递函子时,将类类型对象作为可调用对象传递。通过类类型调用通常相当于调用其函数操作符。因此,

\begin{lstlisting}[style=styleCXX]
op(*current);
\end{lstlisting}

通常会转化为

\begin{lstlisting}[style=styleCXX]
op.operator()(*current); // call operator() with parameter *current for op
\end{lstlisting}

定义函数操作符时,应该将其定义为常量成员函数。否则,当框架或库希望此调用不会改变传递对象的状态时,就会产生错误消息(详见9.4节)。

类类型对象也可以隐式转换为指向代理调用函数的指针或引用(在C.3.5节讨论)。这种情况下,调用

\begin{lstlisting}[style=styleCXX]
op(*current);
\end{lstlisting}

会变成

\begin{lstlisting}[style=styleCXX]
(op.operator F())(*current);
\end{lstlisting}

其中F是类类型对象可以转换为的函数指针或函数引用的类型。

\item
Lambda表达式产生函子(称为闭包),这种情况与函子情况没有区别。然而,Lambda是引入函子的一种非常方便的快捷方式。C++11后,经常出现在C++代码中。

有趣的是,以[]开头的Lambda(没有捕获)产生到函数指针的转换操作符。因为闭包的普通函数操作符会匹配得更好,所以从来没将其作为代理函数。
\end{itemize}

\subsubsubsection{11.1.2\hspace{0.2cm}处理成员函数和附加函数}

前面的例子中没有使用的实体:成员函数。这是因为调用非静态成员函数通常需要使用object.memfunc(…)或ptr->memfunc(…)这样的语法指定调用的对象,而这与通常的模式function-object(…)不匹配。

C++17后,标准库提供了一个实用工具std::invoke(),会将这种情况与普通的函数调用语法统一,从而允许以单一形式调用可调用对象。以下foreach()模板的实现使用了std::invoke():

\hspace*{\fill} \\ %插入空行
\noindent
\textit{basics/foreachinvoke.hpp}
\begin{lstlisting}[style=styleCXX]
#include <utility>
#include <functional>
template<typename Iter, typename Callable, typename... Args>
void foreach (Iter current, Iter end, Callable op, Args const&... args)
{
	while (current != end) { // as long as not reached the end of the elements
		std::invoke(op, // call passed callable with
					args..., // any additional args
					*current); // and the current element
		++current;
	}
}
\end{lstlisting}

这里除了可调用的参数外,还接受任意数量的附加参数。然后,foreach()模板使用给定的可调用对象调用std::invoke(),再加上附加的给定参数和所引用的元素。std::invoke()处理如下:

\begin{itemize}
\item
如果可调用对象是指向成员的指针,则使用第一个附加参数作为this对象。所有附加参数只是作为参数传递给可调用对象。

\item
否则,所有附加参数都只是作为参数传递给可调用对象。
\end{itemize}

不能在这里对可调用参数或附加参数使用完美转发:第一次调用可能“窃取”值,导致在后续迭代中调用op时可能会出现意外行为。

有了这个实现,仍然可以编译上面对foreach()的调用。另外,还可以向可调用对象传递额外的参数,可调用对象也可以是成员函数。

\begin{tcolorbox}[colback=webgreen!5!white,colframe=webgreen!75!black]
\hspace*{0.75cm}std::invoke()允许将指向数据成员的指针作为回调类型。它不调用函数,而是返回附加参数引用的对象中相应数据成员的值。
\end{tcolorbox}

下面的代码说明了这一点:

\hspace*{\fill} \\ %插入空行
\noindent
\textit{basics/foreachinvoke.hpp}
\begin{lstlisting}[style=styleCXX]
#include <iostream>
#include <vector>
#include <string>
#include "foreachinvoke.hpp"

// a class with a member function that shall be called
class MyClass {
	public:
	void memfunc(int i) const {
		std::cout << "MyClass::memfunc() called for: " << i << ’\n’;
	}
};

int main()
{
	std::vector<int> primes = { 2, 3, 5, 7, 11, 13, 17, 19 };
	// pass lambda as callable and an additional argument:
	foreach(primes.begin(), primes.end(), // elements for 2nd arg of lambda
			[](std::string const& prefix, int i) { // lambda to call
				std::cout << prefix << i << ’\n’;
			},
			"- value: "); // 1st arg of lambda

	// call obj.memfunc() for/with each elements in primes passed as argument
	MyClass obj;
	foreach(primes.begin(), primes.end(), // elements used as args
			&MyClass::memfunc, // member function to call
			obj); // object to call memfunc() for
}
\end{lstlisting}

foreach()的第一次调用将其第四个参数(字符串字面量"- value: ")传递给Lambda的第一个参数,而vector中的当前元素绑定到Lambda的第二个参数。第二次调用传递成员函数memfunc()作为传递obj作为第四个参数的第三个参数。

有关产生std::invoke()是否可以使用可调用对象的类型特征,请参见D.3.1节。

\subsubsubsection{11.1.3\hspace{0.2cm}使用包装函数}

std::invoke()的一个常见应用是封装单个函数调用(例如,记录调用,测量持续时间,或准备一些上下文,例如启动一个新线程)。现在,可以通过完美转发可调用参数和传递参数来支持移动语义:

\hspace*{\fill} \\ %插入空行
\noindent
\textit{basics/invoke.hpp}
\begin{lstlisting}[style=styleCXX]
#include <utility> // for std::invoke()
#include <functional> // for std::forward()

template<typename Callable, typename... Args>
decltype(auto) call(Callable&& op, Args&&... args)
{
	return std::invoke(std::forward<Callable>(op), // passed callable with
					   std::forward<Args>(args)...); // any additional args
}
\end{lstlisting}

另一个有趣的地方在于,如何处理调用函数的返回值,以便将其“完美地”转发回调用者。为了支持返回引用(比如std::ostream\&),必须使用decltype(auto)而不是auto:

\begin{lstlisting}[style=styleCXX]
template<typename Callable, typename... Args>
decltype(auto) call(Callable&& op, Args&&... args)
\end{lstlisting}

decltype(auto)(C++14)是一个占位符类型,根据相关表达式的类型(初始化器、返回值或模板参数)确定变量、返回类型或模板参数的类型。详见15.10.3节。

若将std::invoke()返回的值临时存储在变量中,以便在执行其他操作(例如,处理返回值或记录调用结束)后返回,还必须使用decltype(auto)声明临时变量:

\begin{lstlisting}[style=styleCXX]
decltype(auto) ret{std::invoke(std::forward<Callable>(op),
				   std::forward<Args>(args)...)};
...
return ret;
\end{lstlisting}

注意,用auto\&\&声明ret并不正确。作为一个引用,auto\&\&扩展返回值的生命周期直到作用域结束(参见第11.3节),但不超出函数调用者的返回语句。

使用decltype(auto)也有一个问题:若可调用对象的返回类型为void,则不允许将ret初始化为decltype(auto),因为void是一个不完整的类型。现在,有以下选择:

\begin{itemize}
\item
在语句的前一行声明一个对象,其析构函数执行希望实现的可观察行为。例如:

\begin{lstlisting}[style=styleCXX]
struct cleanup {
	~cleanup() {
		... // code to perform on return
	}
} dummy;
return std::invoke(std::forward<Callable>(op),
					std::forward<Args>(args)...);
\end{lstlisting}

\item
以不同的方式实现void和非void的情况:

\hspace*{\fill} \\ %插入空行
\noindent
\textit{basics/invokeret.hpp}
\begin{lstlisting}[style=styleCXX]
#include <utility> // for std::invoke()
#include <functional> // for std::forward()
#include <type_traits> // for std::is_same<> and invoke_result<>

template<typename Callable, typename... Args>
decltype(auto) call(Callable&& op, Args&&... args)
{
	if constexpr(std::is_same_v<std::invoke_result_t<Callable, Args...>,
	void>) {
		// return type is void:
		std::invoke(std::forward<Callable>(op),
					std::forward<Args>(args)...);
		...
		return;
	}
	else {
		// return type is not void:
		decltype(auto) ret{std::invoke(std::forward<Callable>(op),
							std::forward<Args>(args)...)};
		...
		return ret;
	}
}
\end{lstlisting}

\begin{lstlisting}[style=styleCXX]
if constexpr(std::is_same_v<std::invoke_result_t<Callable, Args...>, void>)
\end{lstlisting}

编译时测试调用的返回类型Args…是否有void。有关std::invoke\_result<>的详细信息,请参阅D.3.1节。

\begin{tcolorbox}[colback=webgreen!5!white,colframe=webgreen!75!black]
\hspace*{0.75cm}C++17起可用std::invoke\_result<>。C++11起,要获得返回类型,可以使用:typename std::result\_of<Callable(Args…)>::type
\end{tcolorbox}

\end{itemize}

未来的C++可能会避免对void的特殊处理(参见17.7节)。













如6.1节所示,可以使用转发引用和std::forward<>来“完美转发”泛型参数:

\begin{lstlisting}[style=styleCXX]
template<typename T>
void f (T&& t) // t is forwarding reference
{
	g(std::forward<T>(t)); // perfectly forward passed argument t to g()
}
\end{lstlisting}

然而,有时必须完美地转发不通过参数的泛型代码中的数据。可以使用auto\&\&来创建一个可以转发的变量,假设对函数get()和set()进行了链接调用,其中get()的返回值应该完美地转发给set():

\begin{lstlisting}[style=styleCXX]
template<typename T>
void foo(T x)
{
	set(get(x));
}
\end{lstlisting}

进一步假设需要更新代码,以便对get()产生的中间值执行一些操作。通过将值保存在使用auto\&\&声明的变量中来实现:

\begin{lstlisting}[style=styleCXX]
template<typename T>
void foo(T x)
{
	auto&& val = get(x);
	...
	// perfectly forward the return value of get() to set():
	set(std::forward<decltype(val)>(val));
}
\end{lstlisting}

这样可以避免对中间值的复制。









































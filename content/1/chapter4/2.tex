C++17后,有一个特性可以对参数包的所有参数
使用二元运算符计算结果(初始值可选)。

例如,下面的函数会返回所有入参的和:

\begin{lstlisting}[style=styleCXX]
template<typename... T>
auto foldSum (T... s) {
	return (... + s); // ((s1 + s2) + s3) ...
}
\end{lstlisting}

若参数包为空,表达式通常是错误格式的(除了操作符\&\&的值为true,操作符||的值为false,逗号操作符空参数包的值为void())。

表4.1列出了可能的折叠表达式。

\begin{table}[H]
	\centering
	\begin{tabular}{|l|l|}
		\hline
		\textbf{折叠表达式}                                                                                                      & \textbf{展开}                                                                                                                                                                                                     \\ \hline
		\begin{tabular}[c]{@{}l@{}}( ... op pack )\\ ( pack op ... )\\ ( init op ... op pack )\\ ( pack op ... op init )\end{tabular} & \begin{tabular}[c]{@{}l@{}}((( pack1 op pack2 ) op pack3 ) ... op packN )\\ ( pack1 op ( ... ( packN-1 op packN )))\\ ((( init op pack1 ) op pack2 ) ... op packN )\\ ( pack1 op ( ... ( packN op init )))\end{tabular} \\ \hline
	\end{tabular}
\end{table}

\begin{center}
表4.1. 折叠表达式(C++17)
\end{center}

几乎所有的二元运算符都可以使用折叠表达式(详见12.4.6节)。例如,可以使用折叠表达式来遍历一个二叉树的路径,使用操作符\texttt{->*}:

\hspace*{\fill} \\ %插入空行
\noindent
\textit{basics/foldtraverse.cpp}
\begin{lstlisting}[style=styleCXX]
// define binary tree structure and traverse helpers:
struct Node {
	int value;
	Node* left;
	Node* right;
	Node(int i=0) : value(i), left(nullptr), right(nullptr) {
	}
	...
};

auto left = &Node::left;
auto right = &Node::right;

// traverse tree, using fold expression:
template<typename T, typename... TP>
Node* traverse (T np, TP... paths) {
	return (np ->* ... ->* paths); // np ->* paths1 ->* paths2 ...
}

int main()
{
	// init binary tree structure:
	Node* root = new Node{0};
	root->left = new Node{1};
	root->left->right = new Node{2};
	...
	// traverse binary tree:
	Node* node = traverse(root, left, right);
	...
}
\end{lstlisting}

这里,

\begin{lstlisting}[style=styleCXX]
(np ->* ... ->* paths)
\end{lstlisting}

使用折叠表达式遍历从np开始的可变元素路径。

这种使用初始化器的折叠表达式,可以简化可变参数模板来打印上面的所有参数:

\begin{lstlisting}[style=styleCXX]
template<typename... Types>
void print (Types const&... args)
{
	(std::cout << ... << args) << '\n';
}
\end{lstlisting}

但无法为参数包中的每个元素输出添加打印空格。要添加空格,需要一个额外的类模板,确保参数的输出都会添加一个空格:

\hspace*{\fill} \\ %插入空行
\noindent
\textit{basics/addspace.hpp}
\begin{lstlisting}[style=styleCXX]
template<typename T>
class AddSpace
{
private:
	T const& ref; // refer to argument passed in constructor
public:
	AddSpace(T const& r): ref(r) {
	}
	friend std::ostream& operator<< (std::ostream& os, AddSpace<T> s) {
		return os << s.ref << ' '; // output passed argument and a space
	}
};

template<typename... Args>
void print (Args... args) {
	( std::cout << ... << AddSpace(args) ) << '\n';
}
\end{lstlisting}

表达式AddSpace(args)使用类模板参数推断(参见2.9节)产生AddSpace<Args>(args),这为每个参数创建一个AddSpace对象,该对象引用传递的参数,并在输出表达式时使用该参数添加一个空格。

关于折叠表达式的详细信息请参见12.4.6节。

































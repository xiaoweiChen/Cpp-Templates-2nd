普通的编译错误通常是非常简洁和直指要害的。例如,当编译器说“类X没有‘fun’成员”时,通常不难找出我代码中的问题(例如,可能把run错输入为fun)。模板就不是这样了,来看一些例子。

\hspace*{\fill} \\ %插入空行
\noindent
\textbf{类型不匹配}

考虑以下使用C++标准库例子:

\noindent
\textit{basics/errornovel1.cpp}
\begin{lstlisting}[style=styleCXX]
#include <string>
#include <map>
#include <algorithm>
int main()
{
	std::map<std::string,double> coll;
	...
	// find the first nonempty string in coll:
	auto pos = std::find_if (coll.begin(), coll.end(),
	[] (std::string const& s) {
		return s != "";
	});
}
\end{lstlisting}

包含一个相当小的错误:用于查找集合中第一个匹配字符串的Lambda中,针对给定的字符串进行检查。因为map中的元素是键/值对,所以期望得到std::pair<std::string const,double>。

GNU C++编译器的某个版本会报告以下错误:

{\scriptsize
\begin{tcblisting}{commandshell={}}

1 In file included from /cygdrive/p/gcc/gcc61-include/bits/stl_algobase.h:71:0,
2                  from /cygdrive/p/gcc/gcc61-include/bits/char_traits.h:39,
3                  from /cygdrive/p/gcc/gcc61-include/string:40,
4                  from errornovel1.cpp:1:
5  /cygdrive/p/gcc/gcc61-include/bits/predefined_ops.h: In instantiation of ’bool __gnu_cxx
   ::__ops::_Iter_pred<_Predicate>::operator()(_Iterator) [with _Iterator = std::_Rb_tree_i
   terator<std::pair<const std::__cxx11::basic_string<char>, double> >; _Predicate = main()
   ::<lambda(const string&)>]’:
6  /cygdrive/p/gcc/gcc61-include/bits/stl_algo.h:104:42: required from ’_InputIterator
   std::__find_if(_InputIterator, _InputIterator, _Predicate, std::input_iterator_tag)
   [with _InputIterator = std::_Rb_tree_iterator<std::pair<const std::__cxx11::basic_string
   <char>, double> >; _Predicate = __gnu_cxx::__ops::_Iter_pred<main()::<lambda(const
   string&)> >]’
7  /cygdrive/p/gcc/gcc61-include/bits/stl_algo.h:161:23: required from ’_Iterator std::__
   find_if(_Iterator, _Iterator, _Predicate) [with _Iterator = std::_Rb_tree_iterator<std::
   pair<const std::__cxx11::basic_string<char>, double> >; _Predicate = __gnu_cxx::__ops::_
   Iter_pred<main()::<lambda(const string&)> >]’
8   /cygdrive/p/gcc/gcc61-include/bits/stl_algo.h:3824:28: required from ’_IIter std::find
   _if(_IIter, _IIter, _Predicate) [with _IIter = std::_Rb_tree_iterator<std::pair<const
   std::__cxx11::basic_string<char>, double> >; _Predicate = main()::<lambda(const string&)
   >]’
9   errornovel1.cpp:13:29: required from here
10 /cygdrive/p/gcc/gcc61-include/bits/predefined_ops.h:234:11: error: no match for call to
   ’(main()::<lambda(const string&)>) (std::pair<const std::__cxx11::basic_string<char>,
   double>&)’
11 { return bool(_M_pred(*__it)); }
12              ^~~~~~~~~~~~~~~~~~~~
13 /cygdrive/p/gcc/gcc61-include/bits/predefined_ops.h:234:11: 
   note: candidate: bool (*)(const string&) {aka bool (*)
   (const std::__cxx11::basic_string<char>&)} <conversion>
14 /cygdrive/p/gcc/gcc61-include/bits/predefined_ops.h:234:11: 
   note: candidate expects 2 arguments, 2 provided
15 errornovel1.cpp:11:52: note: candidate: main()::<lambda(
   const string&)>
16                              [] (std::string const& s) {
17                                                             ^
\end{tcblisting}
}
{\scriptsize
\begin{tcblisting}{commandshell={}}
18 errornovel1.cpp:11:52: note: no known conversion for 
   argument 1 from ’std::pair<const std::__cxx11::basic_string<char>, 
   double>’ to ’const string& {aka const std::__cxx11::basic_string
   <char>&}’
\end{tcblisting}
}

这样的信息开始看起来更像一本小说,这可能会让模板新手感到沮丧。通过对这样的消息进行管理,错误会相对更容易定位。

错误消息的第一部分说错误发生在内部predefined\_ops.h头文件中的函数模板实例中,errornovel1.cpp通过其他头文件包含其中。下面几行中,编译器报告用哪个参数实例化了什么。本例中,一切都从errornovel1.cpp的第13行语句开始:

\begin{lstlisting}[style=styleCXX]
auto pos = 
std::find_if (coll.begin(), coll.end(),
	[] (std::string const& s) {
		return s != "";
	});
\end{lstlisting}

这导致在stl\_go.h头文件的第115行实例化了一个find\_if模板

\begin{lstlisting}[style=styleCXX]
_IIter std::find_if(_IIter, _IIter, _Predicate)
\end{lstlisting}

实例化为

\begin{lstlisting}[style=styleCXX]
_IIter = std::_Rb_tree_iterator<std::pair<
	const std::__cxx11::basic_string<char>, double> >
_Predicate = main()::<lambda(const string&)>
\end{lstlisting}

编译器都会进行报告,不期望这些模板实例化。

我们的例子中,所有类型的模板都需要实例化,只是想知道为什么它不工作。这个信息出现在消息的最后一部分:表示“调用不匹配”意味着函数调用无法解析,因为实参类型和形参类型不匹配。

\begin{lstlisting}[style=styleCXX]
(main()::<lambda(const string&)>) (std::pair<const std::__cxx11::basic_string<char>, double>&)
\end{lstlisting}

导致调用代码错误:

\begin{lstlisting}[style=styleCXX]
{ return bool(_M_pred(*__it)); }
\end{lstlisting}

之后,包含“note: candidate:”的行解释了有一个候选类型需要一个const字符串\&,这个候选类型在errornovel1.cpp的第11行定义为Lambda函数[](std::string const\& s),并说明了为什么候选类型不适合:

{\scriptsize
\begin{tcblisting}{commandshell={}}
no known conversion for argument 1
from ’std::pair<const std::__cxx11::basic_string<char>, double>’
to ’const string& {aka const std::__cxx11::basic_string<char>&}’
\end{tcblisting}
}

这也描述了我们的问题。

毫无疑问,错误消息可以呈现的更好。实际问题会在实例化之前发出,而不是使用完全扩展的模板实例化名称,如std::\_\_cxx11::basic\_string<char>,只使用std::string可能就足够了。然而,诊断中的所有信息在某些情况下也可能有用。因此,其他编译器提供类似的信息也就不足为奇了(尽管有些使用了上面提到的结构化技术)。

例如,Visual C++编译器会输出如下错误信息:

{\scriptsize
\begin{tcblisting}{commandshell={}}
1  c:\tools_root\cl\inc\algorithm(166): error C2664: ’bool main::<lambda_b863c1c7cd07048816
   f454330789acb4>::operator ()(const std::string &) const’: cannot convert argument 1 from
   ’std::pair<const _Kty,_Ty>’ to ’const std::string &’
2          with
3          [
4              _Kty=std::string,
5              _Ty=double
6          ]
7  c:\tools_root\cl\inc\algorithm(166): note: Reason: cannot convert from ’std::pair<const
   _Kty,_Ty>’ to ’const std::string’
8          with
9          [
10             _Kty=std::string,
11             _Ty=double
12         ]
13 c:\tools_root\cl\inc\algorithm(166): note: No user-defined-conversion operator available
   that can perform this conversion, or the operator cannot be called
14 c:\tools_root\cl\inc\algorithm(177): note: see reference to function template instantiat
   ion ’_InIt std::_Find_if_unchecked<std::_Tree_unchecked_iterator<_Mytree>,_Pr>(_InIt,_In
   It,_Pr &)’ being compiled
15         with
16         [
17              _InIt=std::_Tree_unchecked_iterator<std::_Tree_val<std::_Tree_simple_types
                <std::pair<const std::string,double>>>>,
18              _Mytree=std::_Tree_val<std::_Tree_simple_types<std::pair<const std::string,
                double>>>,
19              _Pr=main::<lambda_b863c1c7cd07048816f454330789acb4>
20         ]
21 main.cpp(13): note: see reference to function template instantiation ’_InIt std::find_if
   <std::_Tree_iterator<std::_Tree_val<std::_Tree_simple_types<std::pair<const _Kty,_Ty>>>>
   ,main::<lambda_b863c1c7cd07048816f454330789acb4>>(_InIt,_InIt,_Pr)’ being compiled
22         with
23         [
24             _InIt=std::_Tree_iterator<std::_Tree_val<std::_Tree_simple_types<std::pair<
               const std::string,double>>>>,
25             _Kty=std::string,
26             _Ty=double,
27             _Pr=main::<lambda_b863c1c7cd07048816f454330789acb4>
28         ]
\end{tcblisting}
}

再次提供了实例化链,并告诉哪些参数实例化了代码中的哪些位置,我们看到了两次

{\scriptsize
\begin{tcblisting}{commandshell={}}
cannot convert from ’std::pair<const _Kty,_Ty>’ to ’const std::string’
with
[
    _Kty=std::string,
    _Ty=double
]
\end{tcblisting}
}

\hspace*{\fill} \\ %插入空行
\noindent
\textbf{缺少const的编译器}

有时泛型代码的问题只是因为某些编译器。考虑下面的例子:

\hspace*{\fill} \\ %插入空行
\noindent
\textit{basics/errornovel2.cpp}
\begin{lstlisting}[style=styleCXX]
#include <string>
#include <unordered_set>
class Customer
{
private:
	std::string name;
public:
	Customer (std::string const& n)
	: name(n) {
	}
	std::string getName() const {
		return name;
	}
};

int main()
{
	// provide our own hash function:
	struct MyCustomerHash {
		// NOTE: missing const is only an error with g++ and clang:
		std::size_t operator() (Customer const& c) {
			return std::hash<std::string>()(c.getName());
		}
	};

	// and use it for a hash table of Customers:
	std::unordered_set<Customer,MyCustomerHash> coll;
	...
}
\end{lstlisting}

Visual Studio 2013或2015中,这段代码按预期编译。但是,使用g++或clang时,代码编译时会出错。例如,在g++ 6.1上,第一个错误消息如下所示:

{\scriptsize
\begin{tcblisting}{commandshell={}}
1  In file included from /cygdrive/p/gcc/gcc61-include/bits/hashtable.h:35:0,
2                   from /cygdrive/p/gcc/gcc61-include/unordered_set:47,
3                   from errornovel2.cpp:2:
4  /cygdrive/p/gcc/gcc61-include/bits/hashtable_policy.h: In instantiation of ’struct std::
   __detail::__is_noexcept_hash<Customer, main()::MyCustomerHash>’:
5  /cygdrive/p/gcc/gcc61-include/type_traits:143:12: required from ’struct std::__and_<
   std::__is_fast_hash<main()::MyCustomerHash>, std::__detail::__is_noexcept_hash<Customer,
   main()::MyCustomerHash> >’
6  /cygdrive/p/gcc/gcc61-include/type_traits:154:38: required from ’struct std::__not_<
   std::__and_<std::__is_fast_hash<main()::MyCustomerHash>, std::__detail::__is_noexcept_
   hash<Customer, main()::MyCustomerHash> > >’
7  /cygdrive/p/gcc/gcc61-include/bits/unordered_set.h:95:63: required from ’class std::
   unordered_set<Customer, main()::MyCustomerHash>’
8  errornovel2.cpp:28:47: required from here
9  /cygdrive/p/gcc/gcc61-include/bits/hashtable_policy.h:85:34: error: no match for call to
   ’(const main()::MyCustomerHash) (const Customer&)’
10 noexcept(declval<const _Hash&>()(declval<const _Key&>()))>
11              ~~~~~~~~~~~~~~~~~~~~~~~^~~~~~~~~~~~~~~~~~~~~~~~
12 errornovel2.cpp:22:17: note: candidate: std::size_t main()::MyCustomerHash::operator()(
   const Customer&) <near match>
13 std::size_t operator() (const Customer& c) {
14                  ^~~~~~~~
15 errornovel2.cpp:22:17: note: passing ’const main()::MyCustomerHash*’ as ’this’ argument
   discards qualifiers
\end{tcblisting}
}

紧接着是20多条其他错误消息:

{\scriptsize
\begin{tcblisting}{commandshell={}}
16 In file included from /cygdrive/p/gcc/gcc61-include/bits/move.h:57:0,
17                  from /cygdrive/p/gcc/gcc61-include/bits/stl_pair.h:59,
18                  from /cygdrive/p/gcc/gcc61-include/bits/stl_algobase.h:64,
19 from /cygdrive/p/gcc/gcc61-include/bits/char_traits.h:39,
20                  from /cygdrive/p/gcc/gcc61-include/string:40,
21                  from errornovel2.cpp:1:
22 /cygdrive/p/gcc/gcc61-include/type_traits: In instantiation of ’struct std::__not_<std::
   __and_<std::__is_fast_hash<main()::MyCustomerHash>, std::__detail::__is_noexcept_hash<
   Customer, main()::MyCustomerHash> > >’:
23 /cygdrive/p/gcc/gcc61-include/bits/unordered_set.h:95:63: required from ’class std::
   unordered_set<Customer, main()::MyCustomerHash>’
24 errornovel2.cpp:28:47: required from here
25 /cygdrive/p/gcc/gcc61-include/type_traits:154:38: error: ’value’ is not a member of ’std
   ::__and_<std::__is_fast_hash<main()::MyCustomerHash>, std::__detail::__is_noexcept_hash<
   Customer, main()::MyCustomerHash> >’
26      : public integral_constant<bool, !_Pp::value>
27                                                  ^~~~
28 In file included from /cygdrive/p/gcc/gcc61-include/unordered_set:48:0,
29                  from errornovel2.cpp:2:
30 /cygdrive/p/gcc/gcc61-include/bits/unordered_set.h: In instantiation of ’class std::
   unordered_set<Customer, main()::MyCustomerHash>’:
31 errornovel2.cpp:28:47: required from here
\end{tcblisting}
}
{\scriptsize
\begin{tcblisting}{commandshell={}}
32 /cygdrive/p/gcc/gcc61-include/bits/unordered_set.h:95:63: error: ’value’ is not a member
   of ’std::__not_<std::__and_<std::__is_fast_hash<main()::MyCustomerHash>, std::__detail::
   __is_noexcept_hash<Customer, main()::MyCustomerHash> > >’
33        typedef __uset_hashtable<_Value, _Hash, _Pred, _Alloc> _Hashtable;
34                                                                                  ^~~~~~~~~~
35 /cygdrive/p/gcc/gcc61-include/bits/unordered_set.h:102:45: error: ’value’ is not a member
   of ’std::__not_<std::__and_<std::__is_fast_hash<main()::MyCustomerHash>, std::__detail::
   __is_noexcept_hash<Customer, main()::MyCustomerHash> > >’
36 typedef typename _Hashtable::key_type key_type;
37                                                     ^~~~~~~~
...
\end{tcblisting}
}

同样,阅读错误消息也很困难(甚至找到每个消息的开头和结尾都很麻烦)。其要表达的是头文件中hashtable\_policy.h在实例化std::unordered\_set<>中需要

\begin{lstlisting}[style=styleCXX]
std::unordered_set<Customer,MyCustomerHash> coll;
\end{lstlisting}

在实例化中

\begin{lstlisting}[style=styleCXX]
noexcept(declval<const _Hash&>()(declval<const _Key&>()))>
\end{lstlisting}

有匹配的调用

\begin{lstlisting}[style=styleCXX]
const main()::MyCustomerHash (const Customer&)
\end{lstlisting}

(declval<const\_Hash\&>()是main()::MyCustomerHash类型的表达式。)一个可能的候选是

\begin{lstlisting}[style=styleCXX]
std::size_t main()::MyCustomerHash::operator()(const Customer&)
\end{lstlisting}

声明为

\begin{lstlisting}[style=styleCXX]
std::size_t operator() (const Customer& c) {
\end{lstlisting}

错误信息最后说明了这个问题:

{\scriptsize
\begin{tcblisting}{commandshell={}}
passing ’const main()::MyCustomerHash*’ as ’this’ argument discards qualifiers
\end{tcblisting}
}

能看出问题出在哪里吗?

这个std::unordered\_set类模板的实现要求哈希对象的函数调用操作符是const成员函数(参见第11.1.1节);否则,算法内部就会出现错误。

所有其他错误消息从第一个开始级联,并在将const限定符简单地添加到哈希函数操作符时消失:

\begin{lstlisting}[style=styleCXX]
std::size_t operator() (const Customer& c) const {
	...
}
\end{lstlisting}

Clang 3.9在第一个错误消息的末尾给出了更好的提示,哈希函数的operator()没有标记为const:

{\scriptsize
\begin{tcblisting}{commandshell={}}
...
errornovel2.cpp:28:47: note: in instantiation of template class ’std::unordered_set<Customer
, MyCustomerHash, std::equal_to<Customer>, std::allocator<Customer> >’ requested here
std::unordered_set<Customer,MyCustomerHash> coll;
                                                           ^
errornovel2.cpp:22:17: note: candidate function not viable: ’this’ argument has type ’const
MyCustomerHash’, but method is not marked const
std::size_t operator() (const Customer& c) {
                ^
\end{tcblisting}
}

clang在这里提到了默认的模板参数,比如std::allocator<Customer>,而gcc跳过了它们。

使用多个编译器来测试代码是有帮助的。不仅有助于编写更可移植的代码,而且当一个编译器产生特别难以理解的错误消息时,另一个编译器可能会提供易懂的信息。



























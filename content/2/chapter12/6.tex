The general concept and syntax of C++ templates have remained relatively stable since their inception in the late 1980s. Class templates and function templates were part of the initial template facility. So were type parameters and nontype parameters.

However, there were also some significant additions to the original design, mostly driven by the needs of the C++ standard library. Member templates may well be the most fundamental of those additions. Curiously, only member function templates were formally voted into the C++ standard. Member class templates became part of the standard by an editorial oversight.

Friend templates, default template arguments, and template template parameters came afterward during the standardization of C++98. The ability to declare template template parameters is sometimes called higher-order genericity. They were originally introduced to support a certain allocator model in the C++ standard library, but that allocator model was later replaced by one that does not rely on template template parameters. Later, template template  parameters came close to being removed from the language because their specification had remained incomplete until very late in the standardization process for the 1998 standard. Eventually a majority of committee members voted to keep them and their specifications were completed.

Alias templates were introduced as part of the 2011 standard. Alias templates serve the same needs as the oft-requested “typedef templates” feature by making it easy to write a template that is merely a different spelling of an existing class template. The specification (N2258) that made it into the standard was authored by Gabriel Dos Reis and Bjarne Stroustrup; Mat Marcus also contributed to some of the early drafts of that proposal. Gaby also worked out the details of the variable template proposal for C++14 (N3651). Originally, the proposal only intended to support constexpr variables, but that restriction was lifted by the time it was adopted in the draft standard.

Variadic templates were driven by the needs of the C++11 standard library and the Boost libraries (see [Boost]), where C++ template libraries were using increasingly advanced (and convoluted) techniques to provide templates that accept an arbitrary number of template arguments. Doug Gregor, Jaakko Jarvi, Gary Powell, Jens Maurer, and Jason Merrill provided the initial specification for the ¨standard (N2242). Doug also developed the original implementation of the feature (in GNU’s GCC) while the specification was being developed, which much helped the ability to use the feature in the standard library.

Fold expressions were the work of Andrew Sutton and Richard Smith: They were added to C++17 through their paper N4191.
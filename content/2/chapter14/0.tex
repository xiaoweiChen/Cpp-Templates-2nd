Template instantiation is the process that generates types, functions, and variables from generic template definitions.

\begin{tcolorbox}[colback=webgreen!5!white,colframe=webgreen!75!black]
\hspace*{0.75cm}The term instantiation is sometimes also used to refer to the creation of objects from types. In this book, however, it always refers to template instantiation.
\end{tcolorbox}

The concept of instantiation of C++ templates is fundamental but also somewhat intricate. One of the underlying reasons for this intricacy is that the definitions of entities generated by a template are no longer limited to a single location in the source code. The location of the template, the location where the template is used, and the locations where the template arguments are defined all play a role in the meaning of the entity.

In this chapter we explain how we can organize our source code to enable proper template use. In addition, we survey the various methods that are used by the most popular C++ compilers to handle template instantiation. Although all these methods should be semantically equivalent, it is useful to understand basic principles of the compiler’s instantiation strategy. Each mechanism comes with its set of little quirks when building real-life software and, conversely, each influenced the final specifications of standard C++.
Programs are generally constructed using design patterns that map relatively well on the mechanisms offered by a chosen programming language. Because templates introduce a whole new language mechanism, it is not surprising to find that they call for new design elements. We explore these elements in this part of the book. Note that several of them are covered or used by the C++ standard library.

Templates differ from more traditional language constructs in that they allow us to parameterize the types and constants of our code. When combined with (1) partial specialization and (2) recursive instantiation, this leads to a surprising amount of expressive power.

Our presentation aims not only at listing various useful design elements but also at conveying the principles that inspire such designs so that new techniques may be created. Thus, the following chapters illustrate a large number of design techniques, including:

\begin{itemize}
\item 
Advanced polymorphic dispatching

\item 
Generic programming with traits

\item 
Dealing with overloading and inheritance

\item 
Metaprogramming

\item 
 Heterogeneous structures and algorithms

\item 
Expression templates
\end{itemize}

We also present some notes to aid the debugging of templates.
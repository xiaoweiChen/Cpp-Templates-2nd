Polymorphism is the ability to associate different specific behaviors with a single generic notation.

\begin{tcolorbox}[colback=webgreen!5!white,colframe=webgreen!75!black]
\hspace*{0.75cm}Polymorphism literally refers to the condition of having many forms or shapes (from the Greek polymorphos).
\end{tcolorbox}

Polymorphism is also a cornerstone of the object-oriented programming paradigm, which in C++ is supported mainly through class inheritance and virtual functions. Because these mechanisms are (at least in part) handled at run time, we talk about dynamic polymorphism. This is usually what is thought of when talking about plain polymorphism in C++. However, templates also allow us to associate different specific behaviors with a single generic notation, but this association is generally handled at compile time, which we refer to as static polymorphism. In this chapter, we review the two forms of polymorphism and discuss which form is appropriate in which situations. 

Note that Chapter 22 will discuss some ways to deal with polymorphism after introducing and discussing some design issues in between.
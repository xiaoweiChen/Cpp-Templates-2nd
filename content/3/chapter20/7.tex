Tag dispatching has been known in C++ for a long time. It was used in the original implementation of the STL (see [StepanovLeeSTL]), and is often used alongside traits. The use of SFINAE and EnableIf is much newer: The first edition of this book (see [VandevoordeJosuttisTemplates1st]) introduced the term SFINAE and demonstrated its use for detecting the presence of member types (for example).

The “enable if” technique and terminology was first published by Jaakko Jarvi, Jeremiah Will-cock, Howard Hinnant, and Andrew Lumsdaine in [OverloadingProperties], which describes the EnableIf template, how to implement function overloading with EnableIf (and DisableIf) pairs, and how to use EnableIf with class template partial specializations. Since then, EnableIf and  similar techniques have become ubiquitous in the implementation of advanced template libraries, including the C++ standard library. Moreover, the popularity of these techniques motivated the extended SFINAE behavior in C++11 (see Section 15.7 on page 284). Peter Dimov was the first to note that default template arguments for function templates (another C++11 feature) made it possible to use EnableIf in constructor templates without introducing another function parameter.

The concepts language feature (described in Appendix E) is expected in the next C++ standard after C++17. It is expected to make many techniques involving EnableIf largely obsolete. Meanwhile, C++17’s constexpr if statements (see Section 8.5 on page 134 and Section 20.3.3 on page 474) is also gradually eroding their pervasive presence in modern template libraries.
FunctorBridge类模板负责底层函数对象的所有权和操作,实现为一个抽象基类,是FunctionPtr动态多态性的基础:

\hspace*{\fill} \\ %插入空行
\noindent
\textit{bridge/functorbridge.hpp}
\begin{lstlisting}[style=styleCXX]
template<typename R, typename... Args>
class FunctorBridge
{
	public:
	virtual ~FunctorBridge() {
	}
	virtual FunctorBridge* clone() const = 0;
	virtual R invoke(Args... args) const = 0;
};
\end{lstlisting}

FunctorBridge提供了通过虚函数操作存储函数对象所需的基本操作:析构函数、执行复制的clone()操作和调用。操作调用基础函数对象。不要忘记将clone()和invoke()定义为const成员函数。

\begin{tcolorbox}[colback=webgreen!5!white,colframe=webgreen!75!black]
\hspace*{0.75cm}将invoke()设为const,为的是避免通过const函数ptr对象调用非const操作符()重载,这与开发者的预期不符。
\end{tcolorbox}

使用这些虚函数,可以实现FunctionPtr的复制构造函数和函数调用操作符:

\hspace*{\fill} \\ %插入空行
\noindent
\textit{bridge/functionptr-cpinv.hpp}
\begin{lstlisting}[style=styleCXX]
template<typename R, typename... Args>
FunctionPtr<R(Args...)>::FunctionPtr(FunctionPtr const& other)
: bridge(nullptr)
{
	if (other.bridge) {
		bridge = other.bridge->clone();
	}
}

template<typename R, typename... Args>
R FunctionPtr<R(Args...)>::operator()(Args... args) const
{
	return bridge->invoke(std::forward<Args>(args)...);
}
\end{lstlisting}















































类型列表是表示列表的类型,可以由模板元程序操作。提供了与列表相关联的操作:迭代列表中的元素(类型)、添加元素或删除元素。类型列表与大多数运行时数据结构(如std::list)不同,其不允许修改。向std::list中添加一个元素会改变列表本身,而这个改变会让程序中有权访问该列表的操作观察到。另一方面,向类型列表添加一个元素不会改变原始的打字员:向现有的类型列表员添加一个元素会创建一个新的类型列表,而非对原始类型列表的修改。熟悉函数式编程语言(如Scheme、ML和Haskell)的读者可能会感觉到,使用C++中的类型列表和使用这些语言中的列表非常相似。

类型列表通常实现为一个类模板特化,该特化在其模板参数中编码类型列表的内容(即它所包含的类型及其顺序)。类型列表的直接实现对参数包中的元素进行编码:

\hspace*{\fill} \\ %插入空行
\noindent
\textit{typelist/typelist.hpp}
\begin{lstlisting}[style=styleCXX]
template<typename... Elements>
class Typelist
{
};
\end{lstlisting}

类型列表的元素可以直接作为模板参数。空的类型列表表示为<>,只包含int的类型列表表示为<int>,依此类推。下面是一个包含所有带符号整型的类型列表:

\begin{lstlisting}[style=styleCXX]
using SignedIntegralTypes =
			Typelist<signed char, short, int, long, long long>;
\end{lstlisting}

操作这个类型列表通常需要将类型列表拆分成几个部分,通常是将列表中的第一个元素(头)与列表中的其余元素(尾)分开。从Front元函数从类型表中提取第一个元素:

\hspace*{\fill} \\ %插入空行
\noindent
\textit{typelist/typelistfront.hpp}
\begin{lstlisting}[style=styleCXX]
template<typename List>
class FrontT;

template<typename Head, typename... Tail>
class FrontT<Typelist<Head, Tail...>>
{
	public:
	using Type = Head;
};

template<typename List>
using Front = typename FrontT<List>::Type;
\end{lstlisting}

因此,FrontT<SignedIntegralTypes>::Type(简化地写成Front<SignedIntegralTypes>)将生成signed char。类似地,PopFront元函数从列表中删除第一个元素,实现将列表元素分为头和尾,然后从尾中的元素形成一个新的类型列表特化。

\hspace*{\fill} \\ %插入空行
\noindent
\textit{typelist/typelistpopfront.hpp}
\begin{lstlisting}[style=styleCXX]
template<typename List>
class PopFrontT;

template<typename Head, typename... Tail>
class PopFrontT<Typelist<Head, Tail...>> {
	public:
	using Type = Typelist<Tail...>;
};

template<typename List>
using PopFront = typename PopFrontT<List>::Type;
\end{lstlisting}

PopFront<SignedIntegralTypes>会产生类型列表:

\begin{lstlisting}[style=styleCXX]
Typelist<short, int, long, long long>
\end{lstlisting}

还可以将所有元素捕获到模板参数包中,然后创建一个包含所有这些元素的新类型列表特化,从而将元素插入到类型列表的前面:

\hspace*{\fill} \\ %插入空行
\noindent
\textit{typelist/typelistpushfront.hpp}
\begin{lstlisting}[style=styleCXX]
template<typename List, typename NewElement>
class PushFrontT;

template<typename... Elements, typename NewElement>
class PushFrontT<Typelist<Elements...>, NewElement> {
	public:
	using Type = Typelist<NewElement, Elements...>;
};

template<typename List, typename NewElement>
using PushFront = typename PushFrontT<List, NewElement>::Type;
\end{lstlisting}

如我们所期望,

\begin{lstlisting}[style=styleCXX]
PushFront<SignedIntegralTypes, bool>
\end{lstlisting}

生成了:

\begin{lstlisting}[style=styleCXX]
Typelist<bool, signed char, short, int, long, long long>
\end{lstlisting}










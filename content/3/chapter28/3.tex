编写模板时,很难确保模板定义能够针对满足该模板约束的模板参数进行编译。考虑一个简单的find()算法,在数组中查找值,以及其文档约束:

\begin{lstlisting}[style=styleCXX]
// T must be EqualityComparable, meaning:
// two objects of type T can be compared with == and the result converted to bool
template<typename T>
int find(T const* array, int n, T const& value);
\end{lstlisting}

可以想象这个函数模板的简单实现:

\begin{lstlisting}[style=styleCXX]
template<typename T>
int find(T const* array, int n, T const& value) {
	int i = 0;
	while(i != n && array[i] != value)
		++i;
	return i;
}
\end{lstlisting}

这个模板定义有两个问题,当某些在技术上满足模板要求,但行为与模板作者预期略有不同的模板参数时,这两个问题将表现为编译错误。我们将使用原型的概念,根据find()模板指定的需求来测试实现对模板参数的使用。

原型是用户定义的类,可以用作模板参数来测试模板定义是否遵守其对相应模板参数施加的约束。原型是定制的,以满足模板的需求,而不提供任何无关操作。若将原型作为模板参数的模板定义实例化成功,那就知道模板定义不会尝试使用模板没有明确要求的其他操作。

下面是一个原型,用于满足find()算法文档中描述的EqualityComparable概念需求:

\begin{lstlisting}[style=styleCXX]
class EqualityComparableArchetype
{
};

class ConvertibleToBoolArchetype
{
	public:
	operator bool() const;
};

ConvertibleToBoolArchetype
operator==(EqualityComparableArchetype const&,
			EqualityComparableArchetype const&);
\end{lstlisting}

EqualityComparableArchetype没有成员函数或数据,可以提供的唯一操作是重载operator==,以满足find()的相等性要求。operator==本身相当小,仅返回另一个原型ConvertibleToBoolArchetype,其唯一定义的操作是用户定义的到bool的转换。

EqualityComparableArchetype显然满足了find()模板的要求,可以通过使用EqualityComparableArchetype实例化find()来检查find()的实现是否在约定末端有效:

\begin{lstlisting}[style=styleCXX]
template int find(EqualityComparableArchetype const*, int,
EqualityComparableArchetype const&);
\end{lstlisting}

find<equalitycomparablearchetype>的实例化将失败,表明我们已经发现了第一个问题:EqualityComparable描述只需要operator==,但find()的实现依赖于将T对象与!=进行比较。实现可以处理大多数用户定义的类型,可以将==和!=作为一对实现,但实际上这并不正确。原型旨在在模板库开发的早期发现这类问题。

改变find()的实现,使用相等来解决第一个问题,find()模板将成功地(使用原型)编译:

\begin{tcolorbox}[colback=webgreen!5!white,colframe=webgreen!75!black]
\hspace*{0.75cm}程序将编译,但不会链接,因为从未定义重载operator==。这是典型的原型,其通常只是作为编译时检查的辅助工具。
\end{tcolorbox}

\begin{lstlisting}[style=styleCXX]
template<typename T>
int find(T const* array, int n, T const& value) {
	int i = 0;
	while(i != n && !(array[i] == value))
		++i;
	return i;
}
\end{lstlisting}

使用原型展示find()中的第二个问题需要更多的技巧。ind()的新定义现在应用了operator!直接指向operator==的结果。原型中,这依赖于用户定义的到bool的转换和内置的逻辑求反operator!。ConvertibleToBoolArchetype的实现会毒害operator!,使其不能恰当地使用:

\begin{lstlisting}[style=styleCXX]
class ConvertibleToBoolArchetype
{
	public:
	operator bool() const;
	bool operator!() = delete; // logical negation was not explicitly required
};
\end{lstlisting}

可以使用删除函数扩展这个原型

\begin{tcolorbox}[colback=webgreen!5!white,colframe=webgreen!75!black]
\hspace*{0.75cm}删除函数是作为普通函数参与重载解析的函数。但若它们通过重载解析选择,编译器则会产生一个错误。
\end{tcolorbox}

来修改操作符\&\&和||,以发现其他模板定义中的问题。通常,模板实现者希望为模板库中确定的概念开发原型,然后使用这些原型根据其声明的需求测试每个模板的定义。























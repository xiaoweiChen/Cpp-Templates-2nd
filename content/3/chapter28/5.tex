Tracers are relatively simple and effective, but they allow us to trace the execution of templates only for specific input data and for a specific behavior of its related functionality. We may wonder, for example, what conditions must be met by the comparison operator for the sorting algorithm to be meaningful (or correct), but in our example, we have only tested a comparison operator that behaves exactly like less-than for integers.

An extension of tracers is known in some circles as oracles (or run-time analysis oracles). They are tracers that are connected to an inference engine—a program that can remember assertions and reasons about them to infer certain conclusions.

Oracles allow us, in some cases, to verify template algorithms dynamically without fully specifying the substituting template arguments (the oracles are the arguments) or the input data (the inference engine may request some sort of input assumption when it gets stuck). However, the complexity of the algorithms that can be analyzed in this way is still modest (because of the limitations of the inference engines), and the amount of work is considerable. For these reasons, we do not delve into the development of oracles, but the interested reader should examine the publication mentioned in the afternotes (and the references contained therein).
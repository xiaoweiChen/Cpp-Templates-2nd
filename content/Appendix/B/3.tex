With the keyword decltype (introduced in C++11), it is possible to check the value category of any C++ expression. For any expression x, decltype((x)) (note the double parentheses) yields:

\begin{itemize}
\item 
type if x is a prvalue

\item
type\& if x is an lvalue

\item
type\&\& if x is an xvalue
\end{itemize}

The double parentheses in decltype((x)) are needed to avoid producing the declared type of a named entity in case where the expression x does indeed name an entity (in other cases, the parentheses have no effect). For example, if the expression x simply names a variable v, the construct without parentheses becomes decltype(v), which produces the type of the variable v rather than a type reflecting the value category of the expression x referring to that variable.

Thus, using type traits for any expression e, we can check its value category as follows:

\begin{lstlisting}[style=styleCXX]
if constexpr (std::is_lvalue_reference<decltype((e))>::value) {
	std::cout << "expression is lvalue\n";
}
else if constexpr (std::is_rvalue_reference<decltype((e))>::value) {
	std::cout << "expression is xvalue\n";
}
else {
	std::cout << "expression is prvalue\n";
}
\end{lstlisting}

See Section 15.10.2 on page 298 for details.








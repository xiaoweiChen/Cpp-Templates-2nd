

表D.7列出了所有剩余的类型特征,可以查询特殊属性或提供更复杂的类型转换。

\begin{table}[H]
	\begin{center}
	\begin{tabular}{l|l}
		\hline
		\textbf{特性}                                       & \textbf{作用}                                                   \\ \hline
		enable\_if\textless{}B,T=void \textgreater{}         & 当bool B为true时才返回类型T                              \\ \hline
		conditional\textless{}B,T,F \textgreater{}           & 若bool B为真,则返回类型T,否则返回类型F              \\ \hline
		common\_type\textless{}T1,... \textgreater{}         & 所有传递类型的通用类型                                   \\ \hline
		aligned\_storage\textless{}Len \textgreater{}        & 默认Len字节对齐的类型                          \\ \hline
		aligned\_storage\textless{}Len,Align \textgreater{}  & 根据size\_t Align的除数,对齐Len字节的类型 \\ \hline
		aligned\_union\textless{}Len,Types... \textgreater{} & Len字节对齐的Types…联合类型                 \\ \hline
	\end{tabular}
	\end{center}
\end{table}

\begin{center}
表D.7. 剩余类型特征
\end{center}

\textbf{std::enable\_if<cond>::type}

\textbf{std::enable\_if<cond, T>::type}

\begin{itemize}
\item
若cond为true,则在其成员类型中返回void或T。否则,不定义成员类型。

\item
因为当cond为false时,没有定义类型成员,因此该特性可以通常给定的条件,禁用或使用SFINAE退出函数模板。

\item
请参阅第6.3节了解详细信息和第一个例子。有关使用参数包的另一个例子,请参阅第D.6节。

\item
有关std::enable\_if如何实现的详细信息,请参见第20.3节。
\end{itemize}

\textbf{std::conditional<cond, T, F>::type}

\begin{itemize}
\item
若cond为true,则为T,否则为F。

\item
这是19.7.1节介绍的特征IfThenElseT的标准版本。

\item
与普通的C++ if-then-else语句不同,then和else分支的模板参数都是在选择之前进行求值的,因此两个分支都不包含错误的代码,或者程序可能是格式错误的。可能需要添加一个间接级别,以避免then和else分支中的表达式在未使用该分支时进行计算。第19.7.1节演示了特征IfThenElseT的这个功能,其具有相同的行为。

\item
参见第11.5节中的示例。

\item
请参阅19.7.1节,了解std::conditional的实现。
\end{itemize}

\textbf{std::common\_type<T...>::type}

\begin{itemize}
\item
生成给定T1, T2, ..., Tn类型的“共同类型”。

\item
普通类型的计算比本附录中讨论的要复杂一些。粗略地说,当第二个和第三个操作数是U和V类型时(引用类型仅用于确定两个操作数的值类别),U和V两种类型是条件三元操作符产生的类型;若该类型无效,则没有共同类型。decay\_t应用于此结果。这个计算可以通过特化重写为std::common\_type<U, V>(C++标准库中,偏特化存在于持续时间和时间点)。

\item
若没有给出类型或没有公共类型存在,则没有定义类型成员,因此会出错(可能会导致SFINAE输出使用它的函数模板)。

\item
若给出单类型,结果是对该类型为decay\_t。

\item
对于两个以上的类型,common\_type递归地用公共类型替换前两个类型T1和T2。若该进程失败,则没有通用类型。

\item
处理普通类型时,传递的类型是衰变类型,因此特性总是产生衰变类型(参见D.4节)。

\item
参见第1.3.3节对该特性应用的讨论和示例。

\item
这个特性的主模板通常是通过以下方式实现的(这里只使用两个参数):

\begin{lstlisting}[style=styleCXX]
template<typename T1, typename T2>
struct common_type<T1,T2> {
	using type = std::decay_t<decltype(true ? std::declval<T1>()
	: std::declval<T2>())>;
};
\end{lstlisting}
\end{itemize}

\textbf{std::aligned\_union<MIN\_SZ, T...>::type}

\begin{itemize}
\item
生成一个普通的旧数据类型(POD)可用作未初始化存储,大小至少为MIN\_SZ,适合保存给定的类型T1,T2,…Tn。

\item
此外,还可以生成一个静态成员alignment\_value,其值表示所有给定类型中需要严格对齐于该值,对于结果类型来说等价于

\begin{itemize}
\item[-]
std::alignment\_of\_v<type>::value (详见D.3.)

\item[-]
alignof(type)
\end{itemize}

\item
要求至少提供一种类型。

\item
例如:
\begin{lstlisting}[style=styleCXX]
using POD_T = std::aligned_union_t<0, char,
								std::pair<std::string,std::string>>;
std::cout << sizeof(POD_T) << ’\n’;
std::cout << std::aligned_union<0, char,
								std::pair<std::string,std::string>
								>::alignment_value;
		<< ’\n’;
\end{lstlisting}

使用aligned\_union来获取对齐值,而不是使用aligned\_union\_t获取类型。
\end{itemize}

\textbf{std::aligned\_storage<MAX\_TYPE\_SZ>::type}

\textbf{std::aligned\_storage<MAX\_TYPE\_SZ, DEF\_ALIGN>::type}

\begin{itemize}
\item
生成一个普通的旧数据类型(POD)可用作未初始化的存储,其大小可容纳所有可能的类型,其大小最大为MAX\_TYPE\_SZ,并需要考虑到默认对齐或传递DEF\_ALIGN的对齐情况。

\item
要求MAX\_TYPE\_SZ大于0,平台至少要有一种对齐值为DEF\_ALIGN的类型。

\item
例如:
\begin{lstlisting}[style=styleCXX]
using POD_T = std::aligned_storage_t<5>;
\end{lstlisting}
\end{itemize}














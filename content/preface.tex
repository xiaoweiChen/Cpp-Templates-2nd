
\begin{flushright}
\zihao{1} 前言
\end{flushright}

C++中模板的概念已经有30多年的历史了,C++模板在1990年就出现在“带注释的C++参考手册”(ARM;参见[ellisstroustrparm])中了。在此之前,已经在更专业的书籍中描述过。十多年后,我们发现对于这个迷人的、复杂的、强大的C++特性的基本概念和高级技术的文献居然这么少。在这本书的第一版中,我们想要解决这个问题,并决定写一本关于模板的书(可能有点不够谦逊)。

自2002年底发布第一个版本以来,C++发生了很大的变化。C++标准的新迭代增加了新的特性,C++社区的不断创新也展示基于模板的新编程技术。因此,本书的第二版保留了与第一版相同的目标,但是针对的是“现代C++”。

我们以不同的背景和不同的目的来完成写这本书的任务。David(又名“Daveed”)是一位经验丰富的编译器实现者,也是发展核心语言的C++标准委员会工作组的积极参与者,他对模板的所有功能(和问题)的精确和详细的描述很感兴趣。Nico是一名“普通”的应用程序程序员,C++标准委员会库工作组的成员,他对理解模板的所有技术很感兴趣,他可以使用这些技术并从中受益。Doug是一名模板库开发人员,成为编译器实现者和语言设计师后,他对收集、分类和评估用于构建模板库的各种技术很感兴趣。此外,我们希望与读者和整个社区分享这些知识,以减少相应的误解、困惑或忧虑。

因此,将看到日常示例的概念介绍和模板确切行为的详细描述。从模板的基本原则开始,逐步发展到“模板编程的艺术”,将发现(或重新发现)静态多态性、类型特征、元编程和表达式模板等技术。还将更深入地了解C++标准库,其中几乎所有的代码都涉及模板。

在写这本书的过程中,我们学到了很多东西,也获得了很多乐趣。我们希望在阅读时,也能享受这本书带来的东西。